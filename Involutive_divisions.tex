\documentclass[10pt]{easychair}
\usepackage[T1]{fontenc}
\usepackage[utf8]{inputenc}
\usepackage[english]{babel}
\usepackage{amsmath,amssymb,mathrsfs}
\usepackage{tikz}
\usepackage{tikz-cd}
\usepackage{enumitem}

\newtheorem{theorem}{Theorem}[section]
\newtheorem{lemma}[theorem]{Lemma}
\newtheorem{proposition}[theorem]{Proposition}
\newtheorem{corollary}[theorem]{Corollary}

\theoremstyle{definition}
\newtheorem{definition}[theorem]{Definition}
\newtheorem{property}[theorem]{Property}
\newtheorem{remark}[theorem]{Remark}
\newtheorem{example}[theorem]{Example}

% Newcommand

%Titre article
\newcommand\titlePaper{{\color{red}\bf Title}}

% Texte
\newcommand\eg{\emph{e.g.}}
\newcommand\ie{\emph{i.e.}}
\newcommand\G{Gröbner}
\newcommand\todo[1]{{\bf\textcolor{red}{#1}}}
\newcommand\fixmecc[1]{{\textcolor{blue}{{\bf Cyrille:} #1}}}
\newcommand\data{{\color{red}\bf data}}

% Raccourcis mathématiques
\DeclareMathOperator{\supp}{supp}
\newcommand\Span[1]{\langle#1\rangle}
\newcommand\diff[1]{\partial_{#1}}
\newcommand\D{\mathcal{D}}
\DeclareMathOperator{\lm}{lm}
\DeclareMathOperator{\lc}{lc}
\newcommand\basis{\mathscr{B}}
\newcommand\SigmaTheta{\Sigma_{\Theta}}
\newcommand\lambdaB{\lambda_\bullet}

% Structures algébriques
\newcommand\K{\mathbb{K}}
\newcommand\F{\mathscr{F}}
\newcommand\Q{\mathbb{Q}} 
\newcommand\N{\mathbb{N}}
\newcommand\QX{\mathbb{Q}[x_1,\cdots,x_n]}
\newcommand\QXX{\mathbb{Q}(x_1,\cdots,x_n)}
\newcommand\KX{\K X}
\newcommand\Weyl[1]{B_{#1}(\Q)}
\DeclareMathOperator{\Mon}{Mon}
\DeclareMathOperator{\Mul}{Mul}

% Réécriture
\newcommand\rewR{\to_R}
\newcommand\rewS{\twoheadrightarrow_S}
\newcommand\transR{\overset{*}{\to}_R}
\newcommand\transS{\overset{*}{\twoheadrightarrow}_S}
\newcommand\equivR{\overset{*}{\leftrightarrow}_R}
\newcommand\rewTheta{\to_{\Theta}}
\newcommand\divInv[1]{\mid_{#1}}
\DeclareMathOperator{\SNF}{{\it S}-NF}
\DeclareMathOperator{\NF}{NF}

\bibliographystyle{plain}

\begin{document}

\section{Involutive divisions}

In this section, we fix a rational Weyl algebra $\Weyl{n}$, with sets of
indeterminates $X=\{x_1,\cdots,x_n\}$ and
$\Delta=\{\partial_1,\cdots,\partial_n\}$. 
\medskip

We first recall that an {\em involutive division} $L$ on $\Mon(\Delta)$
is given if for any finite set $U\subset\Mon(\Delta)$, a relation
$\divInv{L}^U$ on $U\times\Mon(\Delta)$ is defined such that for every
$u,u'\in U$ and every $m,m'\in\Mon(\Delta)$, we have:
\begin{enumerate}[label=\alph*)]
\item $u\divInv{L}^Um\Rightarrow u\mid m$,
\item $u\divInv{L}^Uu$,
\item $u\divInv{L}^Uum$ and $u\divInv{L}^Uum'$ if and only if
  $u\divInv{L}^Uumm'$,
\item\label{it:vertex} $u\divInv{L}^Um$ and $u'\divInv{L}^Um$ implies
  $u\divInv{L}^Uu'$ or $u'\divInv{L}^Uu$,
\item\label{it:transitivity} $u\divInv{L}^Uu'$ and $u'\divInv{L}^Um$
  implies $u\divInv{L}^Um$,
\item\label{it:filter} for every $V\subseteq U$ and every $v\in V$,
  $v\divInv{L}^Vm$ if and only if $v\divInv{L}^Um$. 
\end{enumerate}
In the sequel, we write $\divInv{L}$ instead if $\divInv{L}^U$ and we say
that $u\in U$ is an {\em L-involutive divisor} of $m\in\Mon(\Theta)$ if
$u\divInv{L}m$.
\medskip

Now, we recall the definitions of cones and completeness. The variable
$\partial_i$ is said to be {\em L-multiplicative} for $u$ w.r.t. $U$ if
$u$ is an $L$-involutive divisor of $u\partial_i$ and the set of
$L$-multiplicative variables for $u$ w.r.t. $U$ is written $\Mul_L(u,U)$.
Notice that an involutive division is entirely determined by the list of
multiplicative variables w.r.t. each finite sets $U$ such that conditions
\ref{it:vertex}, \ref{it:transitivity}, and \ref{it:filter} are
fulfilled. The {\em cone} and the {\em involutive cone} of $U$ w.r.t. to
the involutive division $L$ are the subsets $C(U)$ and $C_L(U)$ of
$\Mon(\Delta)$ defined as follows:
\[C(U):=\bigcup_{u\in U}\{um\mid m\in\Mon(\Delta)\},\quad
C_L(U):=\bigcup_{u\in U}\{um\mid m\in\Mon(\Mul_L(u,U))\}.\]
Finally, we say that $U$ is {\em L-complete} if $C(U)=C_L(U)$.
\smallskip

\begin{definition}
  Let $\Theta\subset\Weyl{n}$ be a finite set of differential operators,
  let $\prec$ be a monomial order, and let $L$ be an involutive division
  on $\Mon(\Delta)$. We say that $\Theta$ is {\em L-complete} if
  $\lm(\Theta)$ is $L$-complete. We say that $\Theta$ is
  {\em involutively L-autoreduced} if for every $\D\in\Theta$, $\lm(\D)$
  is the only element of $\supp(\D)$ which belongs to $C_L(\lm(\Theta))$
  and if it admits only $\lm(\D)$ as $L$-involutive divisor in
  $\lm(\Theta)$.
\end{definition}

\begin{example}
  Thomas, Janet, Pommaret
\end{example}

\begin{proposition}
  Let L be an involutive division on $\Mon(\Delta)$ and let
  $\Theta=\{D_1,\cdots,\D_r\}$ be an autoreduced complete subset of
  $\Weyl{n}$. Then, the set 
  \[S:=\{\lm(\D_i)m_i\rewTheta\frac{r_i}{\lc(\D_i)}m_i\mid1\leq i\leq r,
  \quad m_i\in\Mon(\Mul_L(\lm(\D_i),\lm(\Theta)))\},\]
  is a strategy for $\rewTheta$ w.r.t. the $\prec$-\data.
\end{proposition}

\begin{proof}
  By definition of $<$, the non minimal basis elements of $\Mon(\Delta)$
  are monomials which are reducible by $\rewTheta$. Hence, left-hand
  sides of $S$ are non minimal basis monomials. Conversely, such a
  monomial $m$ is of the form $\lm(\D_j)m_j$, where $1\leq j\leq r$ and
  $m_j$ is an arbitrary monomial, that is, $m$ belongs to the cone
  $C(\lm(\Theta))$. Since $\lm(\Theta)$ is $L$-complete, $C(\lm(\Theta))$
  is equal to the involutive cone $C_L(\lm(\Theta))$, so that there
  exists $1\leq i\leq r$ such that $m=\lm(D_i)m_i$, where $m_i$ contains
  only variables in $\Mul_L(\lm(\D_i),\lm(\Theta))$. Hence, left-hand
  sides of $S$ are exactly non minimal elements of $\Delta$. Moreover,
  since $\Theta$ is $L$-autoreduced, every monomial admits as most one
  $L$-involutive divisor in $\lm(\Theta)$, so that the left-hand sides
  of elements of $S$ are pairwise distinct. Finally,
  $\lm(\D_i)m_i\rewTheta r_i/\lc(\D_i)m_i$ implies
  $r_i/\lc(\D_i)m_i<\lm(\D_i)m_i$. Hence, $S$ is a strategy for
  $\rewTheta$ w.r.t. the $\prec$-data.
\end{proof}

\begin{definition}
  Let $U$ be a finite subset of $\Mon(\Delta)$. An involutive $U$-division is a relation $\divInv{}^U$ satisfying axioms a) to e). The notions of completeness and autoreduced are still well defined in this case.
\end{definition}

\begin{proposition}
Let $\Theta=\{D_1,\cdots,\D_r\}$ be an complete subset of
$\Weyl{n}$,  $U$ its set of leading monomials, and $S$ a strategy for $\rewTheta$.

The following are equivalent:
\begin{itemize}
\item $S$ is the strategy associated to an complete autoreduced involutive $U$-division.
\item The relation $u \divInv{S} v$ defined by $u \divInv{S} v$ if $v = u w$ and $r_v = r_u w$ is an involutive $U$-division. 
\end{itemize}
\end{proposition}

\end{document}

%%% Local Variables:
%%% mode: latex
%%% TeX-master: t
%%% End:
