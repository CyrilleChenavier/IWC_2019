\documentclass[10pt]{easychair}
\usepackage[T1]{fontenc}
\usepackage[utf8]{inputenc}
\usepackage[english]{babel}
\usepackage{amsmath,amssymb,mathrsfs}
\usepackage{tikz}
\usepackage{tikz-cd}
\usepackage{enumitem}

\newtheorem{theorem}{Theorem}[section]
\newtheorem{lemma}[theorem]{Lemma}
\newtheorem{proposition}[theorem]{Proposition}
\newtheorem{corollary}[theorem]{Corollary}

\theoremstyle{definition}
\newtheorem{definition}[theorem]{Definition}
\newtheorem{property}[theorem]{Property}
\newtheorem{remark}[theorem]{Remark}
\newtheorem{example}[theorem]{Example}

% Newcommand

%Titre article
\newcommand\titlePaper{{\color{red}\bf Title}}

% Texte
\newcommand\eg{\emph{e.g.}}
\newcommand\ie{\emph{i.e.}}
\newcommand\G{Gröbner}
\newcommand\todo[1]{{\bf\textcolor{red}{#1}}}
\newcommand\fixmecc[1]{{\textcolor{blue}{{\bf Cyrille:} #1}}}
\newcommand\data{{\color{red}\bf data}}

% Raccourcis mathématiques
\DeclareMathOperator{\supp}{supp}
\newcommand\Span[1]{\langle#1\rangle}
\newcommand\diff[1]{\partial_{#1}}
\newcommand\D{\mathcal{D}}
\DeclareMathOperator{\lm}{lm}
\DeclareMathOperator{\lc}{lc}
\newcommand\basis{\mathscr{B}}
\newcommand\SigmaTheta{\Sigma_{\Theta}}
\newcommand\lambdaB{\lambda_\bullet}

% Structures algébriques
\newcommand\K{\mathbb{K}}
\newcommand\F{\mathscr{F}}
\newcommand\Q{\mathbb{Q}} 
\newcommand\N{\mathbb{N}}
\newcommand\QX{\mathbb{Q}[x_1,\cdots,x_n]}
\newcommand\QXX{\mathbb{Q}(x_1,\cdots,x_n)}
\newcommand\KX{\K X}
\newcommand\Weyl[1]{B_{#1}(\Q)}
\DeclareMathOperator{\Mon}{Mon}
\DeclareMathOperator{\Mul}{Mul}

% Réécriture
\newcommand\rewR{\to_R}
\newcommand\rewS{\twoheadrightarrow_S}
\newcommand\transR{\overset{*}{\to}_R}
\newcommand\transS{\overset{*}{\twoheadrightarrow}_S}
\newcommand\equivR{\overset{*}{\leftrightarrow}_R}
\newcommand\rewTheta{\to_{\Theta}}
\newcommand\divInv[1]{\mid_{#1}}
\DeclareMathOperator{\SNF}{{\it S}-NF}
\DeclareMathOperator{\NF}{NF}

\bibliographystyle{plain}

\begin{document}

\section{Involutive divisions}

In this section, we fix a rational Weyl algebra $\Weyl{n}$, with sets of
indeterminates $X=\{x_1,\cdots,x_n\}$ and
$\Delta=\{\partial_1,\cdots,\partial_n\}$. 
\medskip

We first recall that an {\em involutive division} $L$ on $\Mon(\Delta)$
is given if for any finite set $U\subset\Mon(\Delta)$, a relation
$\divInv{L}^U$ on $U\times\Mon(\Delta)$ is defined such that for every
$u,u'\in U$ and every $m,m'\in\Mon(\Delta)$, we have:
\begin{enumerate}[label=\alph*)]
\item $u\divInv{L}^Um\Rightarrow u\mid m$,
\item $u\divInv{L}^Uu$,
\item $u\divInv{L}^Uum$ and $u\divInv{L}^Uum'$ if and only if
  $u\divInv{L}^Uumm'$,
\item\label{it:vertex} $u\divInv{L}^Um$ and $u'\divInv{L}^Um$ implies
  $u\divInv{L}^Uu'$ or $u'\divInv{L}^Uu$,
\item\label{it:transitivity} $u\divInv{L}^Uu'$ and $u'\divInv{L}^Um$
  implies $u\divInv{L}^Um$,
\item\label{it:filter} for every $V\subseteq U$ and every $v\in V$,
  $v\divInv{L}^Um$ implies $v\divInv{L}^Vm$. 
\end{enumerate}
In the sequel, we write $\divInv{L}$ instead if $\divInv{L}^U$ and we say
that $u\in U$ is an {\em L-involutive divisor} of $m\in\Mon(\Delta)$ if
$u\divInv{L}m$. We say that $U$ is {\em L-autoreduced} if every $u\in U$
admits only $u$ as involutive divisor.
\medskip

Now, we recall the definitions of multiplicative variables, cones, and
completeness. The variable $\partial_i$ is said to be
{\em L-multiplicative} for $u$ w.r.t. $U$ if $u$ is an $L$-involutive
divisor of $u\partial_i$ and the set of $L$-multiplicative variables for
$u$ w.r.t. $U$ is written $\Mul_L(u,U)$. Notice that $u\divInv{L}m$ if
and only if $m=um'$, where $m'\in\Mon(\Mul_L(u,U))$, \ie, $m'$ contains
only monomials in $\Mul_L(u,U)$. Notice also that an involutive division
is entirely determined by the list of multiplicative variables w.r.t.
each finite sets $U$ such that conditions \ref{it:vertex},
\ref{it:transitivity}, and \ref{it:filter} are fulfilled. The {\em cone}
and the {\em involutive cone} of~$U$ w.r.t. to  $L$ are the subsets
$C(U)$ and $C_L(U)$ of $\Mon(\Delta)$ defined as follows:
\[C(U):=\bigcup_{u\in U}\{um: m\in\Mon(\Delta)\},\quad
C_L(U):=\bigcup_{u\in U}\{um: u\in\Mon(\Mul_L(u,U))\}.\]
Notice that $C_L(U)\subseteq C(U)$. Finally, we say that $U$ is
{\em L-complete} if $C_L(U)=C(U)$. 
\smallskip

\begin{definition}
  Let $\Theta\subset\Weyl{n}$ be a finite set of differential operators,
  let $\prec$ be a monomial order, and let $L$ be an involutive division
  on $\Mon(\Delta)$. We say that $\Theta$ is {\em L-complete} and
  {\em L-autoreduced} if $\lm(\Theta)\subset\Mon(\Delta)$ is $L$-complete
  and $L$-autoreduced, respectively.
\end{definition}

\begin{example}
  Thomas, Janet, Pommaret
\end{example}

\begin{proposition}\label{prop:involutive_strategy}
  Let L be an involutive division on $\Mon(\Delta)$ and let
  $\Theta=\{D_1,\cdots,\D_r\}$ be an L-autoreduced and L-complete set of
  differential operators. A strategy for $\rewTheta$ is given by the
  following set:
  \[S:=\{\lm(\D_i)m\rewTheta\frac{1}{\lc(\D_i)}r_im:\ 1\leq i\leq r,
  \quad\lm(\D_i)\divInv{L}\lm(\D_i)m\}.\]
    
  \fixmecc{vérifier l'ordre $\lm(\D_i)m$.}
\end{proposition}

\begin{proof}
  We first have to prove that left-hand sides of $S$ are exactly
  monomials that are not minimal for $<$. This is a consequence of the
  following observations: $1).$ left-hand sides of $S$ form the
  involutive cone $C_L(\lm(\Theta))$, $2).$ since $\Theta$ is complete,
  we have $C_L(\lm(\Theta))=C(\lm(\Theta))$, $3).$ $C(\lm(\Theta))$ is
  the set of monomials that are reducible by $\rewTheta$, $4).$ the set
  of monomials that are reducible by $\rewTheta$ is the set of monomials
  that are not minimal for $<$. Moreover, every monomial~$m$ that is a
  left-hand side of $S$ admits exactly one involutive divisor, so that
  there exists exactly one rewriting rule in $S$ of the form $m\rewS\D$.
  Finally, we have $\D<m$ by definition of $<$, which proves that $S$ is
  a strategy for $\rewTheta$.
\end{proof}

\begin{definition}
  The strategy defined in Proposition~\ref{prop:involutive_strategy} is
  called the {\em L-strategy} w.r.t. $(\Theta,\prec)$.
\end{definition}

\begin{definition}
  Let $\Theta\subset\Weyl{n}$ be a finite set of operators and let $L$ be
  an involutive division on $\Mon(\Delta)$ such that $\Theta$ is
  $L$-autoreduced. We say $\Theta$ is {\em L-involutive} if for every
  $\D\in\Theta$ and every $m\in\Mon(\Delta)$, we have $\SNF(\D m)=0$. We
  say that $\Theta$ is {\em  L-passive} if for every $\D\in\Theta$ and
  every variable $\partial_i$ that is not $L$-multiplicative for
  $\Theta$, we have $\SNF(\D\partial_i)=0$.
\end{definition}



\begin{definition}
  Let $U$ be a finite subset of $\Mon(\Delta)$. An involutive $U$-division is a
  relation $\divInv{}^U$ satisfying axioms a) to e). The notions of completeness
  and autoreduced are still well defined in this case.
\end{definition}

\begin{proposition}
  Let $\Theta=\{D_1,\cdots,\D_r\}$ be an complete subset of $\Weyl{n}$, $U$ its set of
  leading monomials, and $S$ a strategy for $\rewTheta$.
  
  The following are equivalent:
  \begin{itemize}
  \item $S$ is the strategy associated to a complete autoreduced \todo{(not
      really the right condition)} involutive $U$-division.
  \item The relation $u \divInv{S} v$, defined by $u \divInv{S} um$ if either
    $u = v$ or $r_u \neq u$ and $r_{um} = r_u m$, is an involutive $U$-division.
  \end{itemize}
\end{proposition}
\begin{proof}
  Let $S$ be the strategy associated to a complete autoreduced involutive
  $U$-division $L$. Let us show that $\divInv L$ and $\divInv S$ coincide. In
  particular, this will show that $\divInv S$ is an involutive $U$-division.

  Suppose $u \divInv L v$, and let us write $v = uw$. We distinguish two cases:
  \begin{itemize}
  \item If there exists $\D_i$ such that $\lm(\D_i) \divInv L u$, write
    $u = \lm(\D_i)m$, so that $r_u = \lc(\D_i)^{-1} r_i m$. Then we also have
    $\D_i \divInv L v$. Since $v = \lm(\D_i)mw$ we get
    $r_v = \lc(\D_i)^{-1} r_i m w = r_u w$, and thus $u \divInv S v$.
  \item Otherwise, suppose that there exists $\D_i$ such that
    $\lm(\D_i) \divInv L v$. Then either $\lm(\D_i) \divInv L u$ or
    $u \divInv L \lm(\D_i)$.  The first is impossible by hypothesis, the second
    because $L$ is autoreduced.
  \end{itemize}

  Reciprocally, suppose that $u \divInv S v$, with $u \neq v$, and let us write
  $v = uw$. Since $u \divInv v$, $r_u \neq u$ and there exists $\D_i$ such that
  $\lm(\D_i) \divInv L u$. In particular $u = \lm(\D_i)m$ and
  $r_u = \lc(\D_i)^{-1} r_i m$. By hypothesis
  $r_v = r_u w = \lc(\D_i)^{-1} r_i m w$. So
  $\lm(\D_i) \divInv L \lm(\D_i)mw = uw$. So either $u \divInv L uw$ or
  $uw \divInv L u$. The second one is impossible because $uw$ cannot divide $u$,
  and so finally $u \divInv L uw = v$. This completes the proof that
  $\divInv{S} = \divInv{L}$.


  Reciprocally, let $S$ be strategy such that $\divInv{S}$ is an involutive
  $U$-division. Let us show that $\divInv{U}$ is complete and autoreduced. Let
  $v = um \in C(U)$, with $m \in \Mon(\Delta)$, and $u \in U$. Then $r_um < um$, so
  $v$ is not an $S$-normal form, and there exists exists $u' \in U$ and
  $m' \in \Mon(\Delta')$ such that $v = u'm'$ and $r_{v} = r_{u'}m$. This means that
  $u' \divInv S u'm'$, and so $v \in C_S(U)$. This proves that $\divInv S$ is
  complete. \todo{il reste à montrer qu'elle est autoreduced}

  Let now $T$ be the strategy induced by $\divInv S$, and let us show that $S$
  and $T$ agree. Let $u \in \Mon(\Delta)$. We distinguish two cases.
  \begin{itemize}
  \item Suppose first that $u$ is an $S$-normal form. Then the only $v$ such
    that $v \divInv S u$ is $u$ itself. If $u$ is not a $T$-normal form, then
    there exists $\D_i$ such that $\lm(\D_i) \divInv S u$. So $u = \lm(\D_i)$,
    which contradicts the fact that $u$ is an $S$-normal form.
  \item Otherwise, then there exists $\D_i$ such that $u = \lm(\D_i) m$ and
    $r^S_u = \lc(\D_i)^{-1} r_i m$. This implies that
    $\lm(\D_i) \divInv S \lm(\D_i)m$, and thus
    $r^T_u = \lc(\D_i)^{-1} r_i m = r^S_u$.
  \end{itemize}
  

  \todo{there is probably a mistake because I don't think I used the hypothesis on S?}
  
\end{proof}

\end{document}

%%% Local Variables:
%%% mode: latex
%%% TeX-master: t
%%% End:
