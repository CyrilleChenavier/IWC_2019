\documentclass[10pt]{easychair}
\usepackage[T1]{fontenc}
\usepackage[utf8]{inputenc}
\usepackage[english]{babel}
\usepackage{amsmath,amssymb,mathrsfs}
\usepackage{tikz}
\usepackage{tikz-cd}
\usepackage{enumitem}

\newtheorem{theorem}{Theorem}[section]
\newtheorem{lemma}[theorem]{Lemma}
\newtheorem{proposition}[theorem]{Proposition}
\newtheorem{corollary}[theorem]{Corollary}

\theoremstyle{definition}
\newtheorem{definition}[theorem]{Definition}
\newtheorem{property}[theorem]{Property}
\newtheorem{remark}[theorem]{Remark}
\newtheorem{example}[theorem]{Example}

% Newcommand

%Titre article
\newcommand\titlePaper{Strategies in algebraic rewriting, with applications to rational Weyl algebras}

% Texte
\newcommand\eg{\emph{e.g.}}
\newcommand\ie{\emph{i.e.}}
\newcommand\G{Gröbner}
\newcommand\todo[1]{{\bf\textcolor{red}{#1}}}
\newcommand\fixmecc[1]{{\textcolor{blue}{{\bf Cyrille:} #1}}}
\newcommand\data{{\color{red}\bf data}}

% Raccourcis mathématiques
\DeclareMathOperator{\supp}{supp}
\newcommand\Span[1]{\langle#1\rangle}
\newcommand\diff[1]{\partial_{#1}}
\newcommand\D{\mathcal{D}}
\DeclareMathOperator{\lm}{lm}
\DeclareMathOperator{\lc}{lc}
\newcommand\basis{\mathscr{B}}
\newcommand\SigmaTheta{\Sigma_{\Theta}}
\newcommand\lambdaB{\lambda_\bullet}

% Structures algébriques
\newcommand\K{\mathbb{K}}
%\newcommand\F{\mathscr{F}}
\newcommand\Q{\mathbb{Q}}
\newcommand\R{\mathbb{R}} 
\newcommand\N{\mathbb{N}}
\newcommand\QX{\mathbb{Q}[x_1,\cdots,x_n]}
\newcommand\QXX{\mathbb{Q}(x_1,\cdots,x_n)}
\newcommand\KX{\K X}
\newcommand\Weyl[1]{B_{#1}(\Q)}
\newcommand\monBasis{\Mon(\Delta)}
\DeclareMathOperator{\Mon}{Mon}

% Réécriture
\newcommand\rewR{\to_R}
\newcommand\rewRR{\twoheadrightarrow_R}
\newcommand\rewS{\twoheadrightarrow_S}
\newcommand\transR{\overset{*}{\to}_R}
\newcommand\transS{\overset{*}{\twoheadrightarrow}_S}
\newcommand\transTheta{\overset{*}{\to}_\Theta}
\newcommand\equivR{\overset{*}{\leftrightarrow}_R}
\newcommand\rewTheta{\to_\Theta}
\newcommand\rewThetaS[1]{\twoheadrightarrow_{\Theta,#1}}
\newcommand\divInv[1]{\mid_{#1}}
\newcommand\RTheta{R_{\Theta}}
\newcommand\SThetaL{S_{\Theta,L}}
\DeclareMathOperator{\SNF}{{\it S}-NF}
\DeclareMathOperator{\SThetaLNF}{{\it S}_{\Theta,L}-NF}
\newcommand{\SThetaNF}[1]{{\it S}_{\Theta,#1}\operatorname{-NF}}

\bibliographystyle{plain}

\begin{document}

\title{Strategies in algebraic rewriting, \\ with application to rational Weyl algebras}

\author{
Cyrille Chenavier\inst{1}
\and
Maxime Lucas\inst{2}
}

\institute{
  Johannes Kepler University, Institute for Algebra\\
  \email{cyrille.chenavier@jku.at}
\and
Inria Rennes - Bretagne Atlantique, Gallinette team\\
\email{maxime.lucas@inria.fr}
}

\authorrunning{C. Chenavier and M. Lucas}
\titlerunning{\titlePaper}

\maketitle

\begin{abstract}  
  We study the confluence property for rewriting systems whose underlying set of
  terms admits a vector space structure. For that, we use deterministic
  reduction strategies. These strategies are based on the choice of standard
  reductions applied to basis elements.

  We provide a sufficient condition of confluence in terms of the kernel of the
  operator which computes standard normal forms. We present a local criterion to
  check the confluence property in this framework, and show that this criterion
  is related to the diamond lemma for terminating rewriting systems.

  Finally, we relate these strategies to the notion of involutive division on
  rational Weyl algebras, and completely characterise the strategies induced by
  an involutive division.
\end{abstract}

\tableofcontents

\section{Introduction}

Rewriting systems are computational models given by a set of syntactic
expressions and transformation rules used to simplify expressions into
equivalent ones. Since rewriting theory is applicable to different
problems of mathematics and computer science, it was developed for many
syntaxes of terms, \eg, strings, ($\Sigma-$, higher-order, infinitary)
terms, graphs, (commutative, noncommutative, vectors of) polynomials,
(linear combinations of) trees, (higher-dimensional) cells. Abstract
rewriting theory unifies these contexts and provides universal
formulations of rewriting properties, such as termination, normalisation
and (local) confluence. Newman's lemma is one of the most famous results 
of abstract rewriting and asserts that under termination hypothesis, 
local confluence implies confluence.
\medskip

In the context of rewriting over algebraic structures, the Newman's lemma
is used in conjunction with the critical pairs lemma to algorithmically
prove confluence. This is something fundamental since confluent rewriting
systems provide methods for solving decision problems, computing (linear,
homotopy) bases, Hilbert series, or free
resolutions~\cite{MR846601, MR2964639, MR1072284, MR1360005}. From these 
methods, one get constructive proofs of theoretical results, such as 
embedding, coherence or homological theorems~\cite{MR506890, MR0506423,
  MR3347996, MR3742562, MR265437, MR920522}, but also applications to 
problems coming from topics modelled by algebra, such as cryptography, 
analysis of (ordinary differential, partial derivative, time-delay) 
equations or control theory. For instance, many informations of 
functional equations may be read over free resolutions: integrability
conditions, parametrization of solutions, existence of autonomous
curves~\cite{MR2233761, MR1308976}.
\medskip

When one considers algebraic structures with underlying vector spaces
operations, the conjunction of the Newman's lemma and the critical pairs
lemma is traditionally known under the name of diamond's lemma. In 
practice, this lemma is used to test if a generating set of a polynomial
ideal is a \G\ basis, since confluent linear rewriting systems are 
usually induced by \G\ bases or one of their numerous adaptations to
different classes of algebras or operads~\cite{MR506890, MR2202562,
  MR2667136, MR1044911, MR1299371}. As an illustration of such a class,
let us mention Weyl algebras that are models of differential operators
with polynomial coefficients. These algebras are composed of polynomials
over two sets of $n$ variables, the state variables $x_1,\cdots,x_n$ and
the vector field variables $\partial_1,\cdots,\partial_n$, and submitted
to the commutation rules
\[\forall 1\leq i\neq j\leq n:\qquad x_ix_j=x_jx_i,\quad\partial_i
\partial_j=\partial_j\partial_i,\quad\partial_ix_j=x_j\partial_i,\quad
\partial_ix_i=x_i\partial_i+1.\]
These relations represent classical rules from differential calculus: the
second on means that second order derivatives of smooth functions
commute, the third one means that $x_j$ is constant for differentition
with respect to $x_i$ and the last one represents the Leibniz's rule for
differentition with respect to $x_i$, that is,
$\partial_i(x_if)=x_i\partial_i(f)+f$, for any smooth function $f$.
\medskip

\section{A weak version of the diamond lemma}
\label{sec:a_weak_version_of_the_diamond_lemma}

In this section, we introduce rewriting strategies for rewriting systems
on vector spaces and formulate a confluence criterion in terms of
strategies. In particular, we obtain a new proof of the diamond lemma
based on rewriting strategies.
\medskip

Throughout the section, we fix a ground field $\K$, a $\K$-vector space
$V$, and a basis $\basis$ of $V$. We say vectors and basis elements for
elements of $V$ and $\basis$, respectively. Every vector $v$ admits a
unique finite decomposition with respect to the basis $\basis$ and
coefficients in the ground field:

\begin{equation}\label{equ:vector_decompo}
  v=\sum\lambda_ie_i,\quad\lambda_i\neq 0.
\end{equation}
The set of basis elements which appear in the decomposition
\eqref{equ:vector_decompo} is called the {\it support} of $v$ and is
written $\supp(v)$. We also fix a subset $R$ of $\basis\times V$, whose
elements are called rewriting rules. A rewriting rule is denoted by
$e\rewR r$, where $e$ and~$r$ are left and right-hand sides of this rule,
\ie, its images through the natural projections of $R$ on $\basis$ and
$V$, respectively.
\medskip

Throughout the paper, we use the standard terminology of rewriting
theory, see~\cite{MR1629216}. An {\em abstract rewriting system} is a
pair $(A,\to)$, where $A$ is a set and $\to$ is binary relation on $A$,
called {\em rewriting relation}. An element $(a,b)\in\to$ is written
$a\to b$ and is called a {\em reduction}. A {\em normal form} for $\to$
is an element $a\in A$ such that there is no $b\in A$ such that $a\to b$.
The rewriting relation $\to$ is said to be {\em confluent} if whenever
$a$ rewrites into $b$ and $c$, that is, there exist finite sequences of
reductions from $a$ to $b$ and $c$, then there exists $d$ such that $b$
and $c$ rewrite into $d$. 

\subsection{Strategies on linear rewriting systems}
\label{sec:strategies_on_linear_rewriting_systems}

In this section, we present our general framework for rewriting systems
on vector spaces and introduce the notion of strategies for such systems.
These strategies are used to formulate a confluence criterion in 
Section~\ref{sec:confluence_relative_to_a_strategy}.
\medskip

We first extend the rewriting rules into a rewriting relation on $V$ by
considering reductions that rewrite many basis elements at once.
Formally, reductions are of the following form: 

\begin{equation}\label{equ:rewriting_step}
  \sum_{i=1}^n\lambda_ie_i+v\rewR\sum_{i=1}^n\lambda_ir_i+v,
  \smallskip
\end{equation}
where $n$ is a strictly positive integer, $v$ is a vector, and for each
index $i$ in the sum, $\lambda_i$ is a nonzero scalar, $e_i\rewR r_i$ is
a rewriting rule, and $e_i$ does not belong to $\supp(v)$. Notice that we
do not assume that $\supp(v)$ contains only basis elements that are
irreducible for $\rewR$. This rewriting relation is still denoted by
$\rewR$ since it extends the rewriting rules in $R$, and its closures
under transitivity, reflexivity, \todo{scalar multiplication}, and sum
(resp. and symmetry) is denoted by $\transR$ (resp. $\equivR$). Finally,
a normal form for $\rewR$ is called an $R$-normal form. In the sequel, we
refer $\rewR$ as the rewriting relation induced by the rewriting rules
$R$.
\smallskip

\begin{remark}
  Usually, the notations $\overset{*}{\to}$ and
  $\overset{*}{\leftrightarrow}$
  stand for the closures of the binary relation $\to$ under transitivity
  and reflexivity and transitivity, reflexivity, and symmetry. In this
  paper, we also require closures under vector spaces operations since we
  precisely wish to develop a rewriting theory internal to vector spaces.
  \todo{Changer/supprimer la deuxième phrase?}
\end{remark}
\smallskip

\todo{Il faudrait expliquer pourquoi dans le cas d'une pré-stratégie on
  réécrit tout le mondex.}

A {\em pre-strategy} of $R$ is a subset $S$ of $R$ whose left-hand sides 
are pairwise distinct, that is, if $e\rewR r$ and $e\rewR r'$ belong to
$S$, then we have $e\rewR r=e\rewR r'$. Hence,
each $S$-reducible basis element $e$ corresponds to exactly one rewriting
rule $e\rewR r_S(e)$ in~$S$. We extend the pre-strategy $S$ into the
determinitic rewriting relation~$\rewS$\footnote{The double head on the
  arrow marks the difference with the nondeterministic $\to_S$ defined
  such as in~\eqref{equ:rewriting_step}.} defined as follows. Every
vector $u$ admits a unique decomposition 
\[u=\sum_{i=1}^n\lambda_ie_i+v,
\smallskip\]
where each $\lambda_i$ is nonzero, each $e_i$ is a left-hand side of a
rule $e_i\rewR r_S(e_i)$ in $S$, and each basis element in $\supp(v)$ is
not. If $n$ is nonzero, then we let
\[u\rewS\sum_{i=1}^n\lambda_ir_S(e_i)+v.
\smallskip\]
The closure of $\rewS$ under transitivity and symmetry is denoted by
$\transS$. A normal form for $\rewS$ is called an $S$-normal form. Notice
that $u\transS v$ implies $u\transR v$ and that $R$ is a pre-strategy for
itself only if its left-hand side are pairwise distinct. 
\smallskip

\begin{definition}
  A pre-strategy is called a {\em strategy} if the rewriting relation
  $\rewS$ is terminating. 
\end{definition}
\smallskip

Following our convention for elements of $R$, those of $S$ are written in
the form $e\rewS r_S(e)$, and the rewriting relation $\rewS$ is called
the rewriting (pre-)strategy induced by $S$. In order to illustrate our
notions, we consider the following running example.
\smallskip

\begin{example}\label{ex:h_norma_form}
  Let $\basis:=\{a,b,c,d\}$ and consider the set $R$ of rewriting rules
  defined as follows:
  \[a\rewR b,\quad b\rewR c+d,\quad c\rewR b-d.
  \smallskip\]
  Note that $\rewR$ is not terminating since there is a rewriting loop
  due to $b\rewR c+d$ and $c+d\rewR (b-d)+d=b$. We select the
  pre-strategy $S$ composed of the following rules:
  \[a\rewS b,\quad b\rewS c+d.
  \smallskip\]
  This pre-strategy is clearly a strategy since every vector admits an
  $S$-normal form after one reduction by $\rewS$. The $R$-normal forms
  are the elements of the form $\lambda_dd$, while the $S$-normal forms
  are all the expressions of the form $\lambda_dd+\lambda_cc$.
\end{example}

\begin{example}\label{ex:case_N}
  Let $\basis = \mathbb N$, with set of rewriting rules $R$ given by
  $n \rewR n+1$. Then a pre-strategy $S$ corresponds to a subset $E$ of
  $\mathbb N$. It is a strategy if and only if for all $n \in E$, there
  exists $k \in \mathbb N$ such that $n + k \notin E$. In other words $S$
  is a strategy if and only if the complement of $E$ in $\mathbb N$ is
  cofinal in $\mathbb N$. \todo{Expliquer ce que veut dire cofinal}
\end{example}
\smallskip

We finish this section by introducing the normalisation operator of a
strategy $S$. By definition of a strategy, for every $v\in V$, there
exists at most one $v'$ such that $v\rewS v'$. Moreover, since~$\rewS$ is
terminating, $v$ is sent by multiple applications of~$\rewS$ to a unique
$S$-normal form, written $\SNF(v)$. This defines a map $\SNF:V\to V$ that
we use to introduce our confluence criterion in
Section~\ref{sec:confluence_relative_to_a_strategy}.
\medskip

\begin{proposition}\label{prop:linearity_of_H}
  The map $\SNF$ is a linear projector.
\end{proposition}

\begin{proof}
  The $S$-normal forms are closed under sum, so that
  $\SNF(\SNF(v))=\SNF(v)$ for every vector $v$, so that $\SNF$ is a
  projector. Moreover, if $u\rewS u'$ and $v\rewS v'$, then by definition
  of~$\rewS$, we have $u+v\rewS u'+v'$. Iterating~$\rewS$, we get
  \[\SNF(u+v)=\SNF(\SNF(u)+\SNF(v))=\SNF(u)+\SNF(v),
  \smallskip\]
  which proves linearity of $\SNF$.
\end{proof}

\subsection{Confluence relative to a strategy}
\label{sec:confluence_relative_to_a_strategy}

In this section, we introduce $S$-confluence and show that this property
implies confluence of~$\rewR$. We also show that $S$-confluence is
characterised in terms of a decreasingness property. We finish by a new
proof of the diamond lemma, based on $S$-confluence.
\smallskip

\begin{definition}\label{def:standardisation_property}
  Given a strategy $S$ for $R$, we say that $\rewR$ is \emph{S-confluent}
  if for every rewriting rule~$e\rewR r$, we have $\SNF(e-r)=0$.
\end{definition}
\smallskip

In the following theorem, we show that under $S$-confluence hypothesis,
$\rewR$ is confluent and the equivalence relation $\equivR$ that it
induces is entirely characterised by $S$-normal forms. 
\medskip

\begin{theorem}\label{thm:S-confluence_criterion}
  Let $R$ be a set of rewriting rules and let $S$ be a strategy for $R$.
  If $\rewR$ is $S$-confluent, then we have $u\equivR v$ if and only if
  $\SNF(u-v)=0$. Moreover, $\rewR$ is confluent.
\end{theorem}

\begin{proof}
  We show the first assertion. The relation $\equivR$ is the closure of
  $\rewR$ under transitivity, symmetry, scalar multiplication and sum.
  Since the relation $\SNF(u-v)=0$ is closed under these operations and
  contains $\rewR$ by $S$-confluence, we get one implication.
  Reciprocally, if $\SNF(u-v)=0$, then we have $\SNF(u)=\SNF(v)$, and by
  definition of $\SNF$, we get $u\transS\SNF(u)$ and $v\transS\SNF(v)$.
  Finally, $v_1\transS v_2$ implies $v_1\transR v_2$ and we get
  $u\equivR v$.

  For the second assertion, let us consider three vectors
  $v,v_1,v_2\in V$ such that for $i=1,2$, we have $v\transR v_i$. By the
  first part of the proposition, we have $\SNF(v_1)=\SNF(v_2)$. Since we
  have $v_i\transS\SNF(v_i)$, we get $v_i\transR\SNF(v_i)$, so that
  $\rewR$ is confluent.
\end{proof}
\smallskip

Note that $S$-confluence is a sufficient but not a necessary condition for
confluence. Indeed, with $\basis$ the set of integers and the rewriting rules
$n\rewR n+1$ as in Example~\ref{ex:case_N}, there is no strategy such
that~$\rewR$ is confluent relative to this strategy.  \smallskip

\begin{example}\label{ex:S-conf}
  Let us continue Example~\ref{ex:h_norma_form}. The following identities
  hold:
  \[\SNF(a)=c+d=\SNF(b),\qquad\SNF(b)=c+d=\SNF(c+d),\qquad
  \SNF(c)=c=\SNF(b-d),
  \smallskip\]
  so that $\rewR$ is $S$-confluent, and hence confluent. Notice that if
  we replace the rule $c\rewR b-d$ by $c\rewR b$, we get $\SNF(c)=c$ and
  $\SNF(b)=c+d$, so $\rewR$ is not $S$-confluent anymore. 
\end{example}
\smallskip

In Proposition~\ref{prop:S-conf_decreasing}, we show that $S$-confluence
is caracterised in terms of the decreasingness property that we introduce
in Definition~\ref{def:decreasing}. In order to state this definition, we
introduce the following notation: for every basis element $e$, we write
$e\overset{=}{\twoheadrightarrow}_Sr_S(e)$, for $e\rewS r_S(e)$ if such a
rewriting rule exists in the strategy $S$ and $e=r_S(e)$, otherwise.
\smallskip

\begin{definition}\label{def:decreasing}
  Given a strategy $S$ for $R$ and a well-founded order $<$ on $R$,
  we say that $R$ is {\em decreasing} w.r.t.\ $(S,<)$ if for every
  rewriting rule $e\rewR r$, we have a diagram:
  \[\begin{tikzcd}
      e\ar[d, "_R"']\ar[r, twoheadrightarrow, "=", "_S"'] &
      r_S(e)\ar[d, leftrightarrow, dotted, "*"', "_R"]\\
      r\ar[r, leftrightarrow, dotted, "*", "_R"'] & v
    \end{tikzcd}\]
  where each rewriting rule occurring in the dotted arrows is strictly
  smaller than the rewriting rule $e\rewR r$ relative to $<$.
\end{definition}
\smallskip

\begin{proposition}\label{prop:S-conf_decreasing}
  Let $S$ be a strategy for $R$. The following assertions are equivalent.
  \begin{enumerate}
  \item $R$ is $S$-confluent.
  \item There exists a well-founded order $<$ on $R$ such that $R$ is
    decreasing w.r.t.\ $(S,<)$.
  \end{enumerate}
\end{proposition}

\begin{proof}
  $(1)\Rightarrow (2)$: We define the order $<$ on $R$ by
  $e\rewS r_S(e)<e'\rewR r$ whenever $e'\rewR r\notin S$. This order is
  clearly well-founded since each chain of strictly decreasing elements
  has length $2$. Let $e\rewR r$ be a rewriting rule and let us construct
  a decreasing diagram as in Definition~\ref{def:decreasing}. We
  distinguish three cases.
  \begin{itemize}
  \item If $e$ is an $S$-normal form, then we have $\SNF(e)=e$ and
    $e\rewR r$ does not belong to~$S$. Moreover, since $\rewR$ is
    $S$-confluent, we have $\SNF(e)=\SNF(r)$, so that we get the
    following diagram
    \[\begin{tikzcd}
    e\ar[d, "_R"']\ar[r, equal] &
    e\ar[d, equal]\\
    r\ar[r, twoheadrightarrow, "*", "_S"'] & e
    \end{tikzcd}\]
    Since $r\transS e$ implies $r\transR e$ and since $e\rewR r$ is greater
    than all elements of $S$ by definition of $\prec$, this diagram is
    decreasing w.r.t.\ $(S,<)$.
  \item If $e\rewR r=e\rewS r_S(e)$ belongs to $S$, then we have the 
    following diagram:
    \[\begin{tikzcd}
    e\ar[r, twoheadrightarrow, "_S"']\ar[d, "_R"']
    & r_S(e) \ar[equal, d]\\
    r_S(e)\ar[r, equal] & r_S(e)
    \end{tikzcd}\]
    Since no rewriting step occurs in the right and bottom faces, this
    diagram is decreasing w.r.t.\ $(S,<)$.
  \item In the last case, $e$ is not a $S$-normal form and $e\rewR r$
    does not belong to $S$. In particular, there exists a rewriting rule
    of the form $e\rewS r_S(e)$ in the strategy $S$. Moreover, by
    definitions of the map $\SNF$ and of the $S$-confluence property, we
    have the equalities
    \[\SNF(r_S(e))=\SNF(e)=\SNF(r).
    \smallskip\]
    Hence, we get the following diagram:
    \[\begin{tikzcd}
    e\ar[r, "_R"']\ar[d, twoheadrightarrow, "_S"'] &
    r \ar[twoheadrightarrow, , "_S", "*"', d] \\
    r_S(e) \ar[r, twoheadrightarrow, "_S"', "*"] & \SNF (e)
    \end{tikzcd}\]
    Since $e\rewR r$ does not belong to $S$ and since every rewriting
    step in the right and bottom faces belong to $S$, they are smaller
    than $e\rewR r$ for $<$ by definition of this order. Hence, the
    diagram is decreasing w.r.t.\ $(S,<)$.
  \end{itemize}
  $(2)\Rightarrow (1)$: Let $e\rewR r$ be a rewriting rule and let us
  assume by induction that for every rewriting step $e'\rewR r'$ smaller
  than $e\rewR r$ for $<$, we have $\SNF(e')=\SNF(r')$. Consider a
  decreasing diagram:
  \[\begin{tikzcd}
  e\ar[d, "_R"']\ar[r, twoheadrightarrow, "=", "_S"'] &
  r_S(e)\ar[d, leftrightarrow, dotted, "*"', "_R"]\\
  r\ar[r, leftrightarrow, dotted, "*", "_R"'] & v
  \end{tikzcd}\]
  Using our induction hypothesis and adapting the argument in the first
  part of the proof of Theorem~\ref{thm:S-confluence_criterion}, we have
  $\SNF(r)=\SNF(v)=\SNF(r_S(e))$. Hence, since $\SNF(e)=\SNF(r_S(e))$ by
  definition of the map $\SNF$, we have $\SNF(e)=\SNF(r)$. The order
  $<$ being well-founded, this inductive argument proves that
  $\SNF(e)=\SNF(r)$ for every rewriting rule $e\rewR r$, that is, $R$ is
  $S$-confluent.
\end{proof}
\smallskip

\begin{remark}
  In the case where the rewriting system comes from a set-theoretic rewriting
  system (that is, the right hand sides of the rewriting rules are elements of
  the basis), the fact that local $S$-confluence implies that $\rewR$ is
  confluent is a special case of Van Ostroom's decreasing diagrams
  \cite{van2008confluence}. More precisely, based on
  Proposition~\ref{prop:S-conf_decreasing}, local $S$-confluence implies
  that the pair of rewriting relations $(\rewS,\rewR)$ is decreasing with
  respect to conversions (see~\cite[Definition 3]{van2008confluence}),
  using the order $<$ on $R$ and the discrete order on $\rewS$.
  By~\cite[Theorem 3]{van2008confluence}, this implies that
  $(\rewS,\rewR)$ commutes. Using the fact that $\rewS \subseteq \rewR$,
  one can then recover that $\rewR$ is confluent.
\end{remark}
\smallskip

\begin{example}\label{ex:end_to_example}
  Let us illustrate Proposition~\ref{prop:S-conf_decreasing} with
  Example~\ref{ex:S-conf}. Let us consider the following order $<$ on
  rewriting rules:
  \[(a\rewR b)<(c\rewR b-d),\quad(b\rewR c+d)<(c\rewR b-d).
  \smallskip\]
  This choice is guided by the heuristic that rules advancing towards an
  $S$-normal form should be favored over rules that do not: here $c$ is
  an $S$-normal form, so the rule that rewrites it should be larger for
  the order $<$. The decreasing diagrams are the following:
  \[\begin{tikzcd}
  a\ar[d, "_R"']\ar[r, twoheadrightarrow, "_S"'] &
  b\ar[d, equal] & b\ar[d, "_R"']
  \ar[r, twoheadrightarrow, "_S"'] & c+d\ar[d, equal] &
  c\ar[d, "_R"']\ar[r, twoheadrightarrow, "_S"'] & c\ar[d,equal]\\
  b\ar[r, equal] & b & c+d\ar[r, equal] & c+d & b-d\ar[r,  "_R"'] & c
  \end{tikzcd}\]
\end{example}
\smallskip

We finish this section by showing how the diamond lemma fits as a
particular case of our setup.
\medskip

\begin{theorem}[\cite{MR506890}]\label{thm:diamond_lemma}
  Let $R$ be a set of rewriting rules such that $\rewR$ is terminating
  and for every $e\in\basis$ such that $e\rewR r$ and $e\rewR r'$, $r$
  and $r'$ are joinable. Then,~$\rewR$ is confluent.
\end{theorem}

\begin{proof}
  For every basis element $e$ that is reducible by $\rewR$, we select
  exactly one arbitrary rewriting rule with left hand-side
  $e$. Then, let $S$ be the pre-strategy composed of these selected
  rewriting rules. Since $\rewR$ is terminating and contains $\rewS$, the
  latter is also terminating, so that $S$ is a strategy for $R$. Let us
  show that $\rewR$ is $S$-confluent using the criterion of
  Proposition~\ref{prop:S-conf_decreasing}. For that, we define the order
  $<$ on $R$ by letting $(e\rewR r)<(e'\rewR r')$ whenever there exists a
  vector $v$ such that $e'\transR v$ and $e\in\supp(v)$. This order is
  well-founded since $\rewR$ is terminating. Now, let $e\rewR r$ be a
  rewriting rule, so that $e$ is not an $R$-normal form and
  $e\overset{=}{\rewS}r_S(e)$ means $e\rewS r_S(e)$. Using the confluence
  hypothesis of the theorem, we have a diagram
  \[\begin{tikzcd}
  e\ar[d, "_R"']\ar[r, twoheadrightarrow, "_S"'] &
  r_S(e)\ar[d, rightarrow, "*"', "_R"]\\
  r\ar[r, rightarrow, "*", "_R"'] & v
  \end{tikzcd}\]
  Each rewriting rule appearing in the right and bottom faces is strictly
  smaller than $e\rewR r$ for~$<$ by definition of this order. Hence,
  this diagram is decreasing, so that $R$ is $S$-confluent. From 
  Theorem~\ref{thm:S-confluence_criterion}, $\rewR$ is confluent.   
\end{proof}
\smallskip

Notice that in the proof of the diamond lemma, we select for $<$ another order
than the one given in the proof of Proposition~\ref{prop:S-conf_decreasing} (the
latter asserts that each rule of $S$ is smaller than each rule which is not in
$S$ and there is no other comparison). This is a good illustration of the
flexibility of the characterisation of $S$-confluence given in
Proposition~\ref{prop:S-conf_decreasing}.

% That illustrates that even if $S$-confluence seems to
% be a strict notion at a first sight, it is flexible in the sense that
% allowing to reverse rewriting rules does not bring more confluence
% diagrams as soon as these rules are small enough for at least one
% well-founded order. 

\section{Differential elimination in rational Weyl algebras}
\label{sec:differential_elimination_in_rational_Weyl_algebras}

In this section, we introduce rewriting systems on rationnal Weyl
algebras and relate involutive divisions to rewriting strategies for such
systems. In particular, we show that involutive sets in rationnal Weyl
algebras induce confluent rewriting systems.
\medskip

Throughout the section, we fix a set $X=\{x_1,\cdots,x_n\}$ of
indeterminates and we denote by $\Q(X):=\QXX$ the field of fractions of
the polynomial algebra $\QX$ over~$\Q$. We fix another set of variables
$\Delta=\{\diff{1},\cdots,\diff{n}\}$ that model partial derivative
operators, see Example~\ref{ex:diff_operators_init}. We denote by
$\partial^{\alpha}:=\diff{1}^{\alpha_1}\cdots\diff{n}^{\alpha_n}$ the
monomial over $\Delta$ with multi-exponent
$\alpha=(\alpha_1,\cdots,\alpha_n)\in\N^n$. Finally, let $\monBasis$ be
the set of monomials over $\Delta$:
\[\monBasis:=\left\{\partial^\alpha:\ \alpha\in\N^n\right\}.
\smallskip\]
In what follows, we keep the terminology monomials only for elements of
$\Mon(\Delta)$ and not for elements in $\Mon(X)$.

\subsection{Rewriting systems on rational Weyl algebras}
\label{sec:rewriting_systems_on_Weyl_algebras}

In this section, we recall the definition of the rational Weyl algebra
and introduce rewriting systems on the rational Weyl algebra induced by
monic operators.
\medskip

\begin{definition}
  The {\it rational Weyl algebra} over $\Q(X)$ is the set of polynomials
  $\Q(X)[\Delta]$ with coefficients in $\QXX$ and indeterminates
  $\Delta$. The multiplication of this $\mathbb Q$-algebra is induced by the
  commutation laws $\partial_i\partial_j=\partial_j\partial_i$ and
  \[\diff{i}f=f\diff{i}+\frac{d}{dx_i}(f),\quad f\in\Q(X),\quad
  1\leq i\leq n,
  \smallskip\]
  where $d/dx_i:\Q(X)\to\Q(X)$ is the partial derivative operator with
  respect to~$x_i$. This algebra is denoted by $\Weyl{n}$.
\end{definition}
\smallskip

Notice that $\Weyl{n}$ is a $\QXX$-vector space and that the monomial set
$\monBasis$ is a basis of~$\Weyl{n}$. Elements of~$\Weyl{n}$ should be
thought of as differential operators with rational functions coefficients
and for this reason, a generic element of this algebra is denoted by $\D$
and is called a differential operator. In the following example, we
illustrate how these operators provide an algebraic model of linear
systems of ordinary differential (in the case $n=1$) and partial
derivative equations (in the case $n\geq 2$). 
\smallskip

\begin{example}\label{ex:diff_operators_init}
  {\color{white}toto}
  \begin{enumerate}
  \item\label{it:ODE_init} The linear ordinary differential equation
    $y'(x)=xy(x)$ is written in the form $(\D y)(x)=0$, where the operator
    $\D:=\partial-x$ belongs to $\Weyl{1}=\Q(x)[\partial]$. 
  \item\label{it:Janet_example_init} Consider Janet's
    example~\cite{MR1308976}, that is, the linear system of partial
    derivative equations with~$3$ variables, one unknown function, and
    the two equations $y_{33}(x)=x_2y_{11}(x)$ and $y_{22}(x)=0$, where
    $y_{ij}(x)$ denotes the second order derivative of the unknown
    function $y(x)$ with respect to the variables $x_i$ and~$x_j$. Then,
    these equations are written $(\D_1y)(x)=0$ and $(\D_2y)(x)=0$, where
    $\D_1,\D_2\in\Weyl{3}$ are defined as follows:
    \[\D_1:=\partial_3^2-x_2\partial_1^2,\quad \D_2:=\partial_2^2.
    \smallskip\]
  \end{enumerate}
\end{example}

\begin{remark}
  In~\ref{it:ODE_init} of Example~\ref{ex:diff_operators_init}, we
  implicitly used that every $f\in\QXX$ induces a unique multiplication
  operator $y(x)\mapsto f(x)y(x)$.
\end{remark}
\smallskip

The next step before introducing rewriting systems on rationanl Weyl
algebras is to recall the definition of monic operators. We fix a
monomial order $\prec$ on $\monBasis$, that is, a well-founded total 
order which is admissible, \ie, $\partial^{\alpha}\prec\partial^{\beta}$
implies $\partial^{\alpha+\gamma}\prec\partial^{\beta+\gamma}$, for every
$\alpha,\beta,\gamma\in\N^n$. Given an operator $\D$, we denote by
$\lm(\D)$ the leading monomial of $\D$ with respect to~$\prec$, that 
is, $\lm(\D)$ is the greatest element of $\supp(\D)$, where the support 
is defined w.r.t.\ the basis $\monBasis$. 
\smallskip

\begin{definition}
  Let $\prec$ be a monomial order on $\monBasis$. A differential
  operator $\D\in\Weyl{n}$ is said to be $\prec$-{\em monic} if the
  coefficient of $\lm(\D)$ on $\D$ is equal to $1$. Moreover, given a
  monic differential operator $\D$, we denote by $r(\D):=\lm(\D)-\D$.
\end{definition}
\smallskip

Since the monomial order $\prec$ is fixed, me simply say monic instead of
$\prec$-monic. Given a set~$\Theta\subseteq\Weyl{n}$ of monic operators,
let us consider the rewriting relation $\rewTheta$ induced by the
following rewriting rules: 
\begin{equation}\label{equ:rewTheta}
  \RTheta:=\Big\{\partial^\alpha\lm(\D)\to_{\RTheta}\partial^\alpha
  r(\D):\ \D\in\Theta,\ \partial^\alpha\in\Mon(\Delta)\Big\}.
\end{equation}
For simplicity, we write $\D\rewTheta\D'$ instead of
$\D\to_{R_\Theta}\D'$. The rewriting relation $\rewTheta$ is terminating
since the rewriting rules reduce a monomial into a combination of
strictly smaller monomials w.r.t.\ the well-founded order $\prec$.
Moreover, notice that in the case where the coefficient $\lc(\D)\in\QXX$
of $\lm(\D)$ in $\D$ is not constant, the situation is much harder.
Indeed, in this case, the left-hand sides of the rewriting rules are of
the form $\partial^\alpha(\lc(\D)\lm(\D))$ and due to commutation laws,
these elements are not monomials. In particular, we are not in the
situation of our general approach developped in  
Section~\ref{sec:a_weak_version_of_the_diamond_lemma} anymore.
\medskip

We finish this section with some comments on $\rewTheta$. Let us consider 
the linear system of ordinary differential or partial derivative
equations given by 
\begin{equation}\label{equ:PDE_system}
  \{(\D y)=0:\D\in\Theta\}.
\end{equation}
Let $y(x)$ be an arbitrary solution to this system. Then, for every
operator $\partial^\alpha$ and every $\D\in\Theta$, we also have
$(\partial^\alpha\D y)(x)=0$, or equivalently,
$(\partial^\alpha\lm(\D)y)(x)=(\partial^\alpha r(\D)y)(x)$. Hence, if
there is a rewriting path $\D_1\transTheta\D_2$, then the solution $y(x)$
of~\eqref{equ:PDE_system} satisfies $(\D_1y)(x)=(\D_2y)(x)$. This remark
has deep applications in the formal theory of partial differential
equations, for instance for finding integrability conditions or computing
dimensions of solution spaces, see~\cite{MR1308976}. Moreover, notice
that since $\monBasis$ is a commutative set, there is another possible
choice for rewriting the monomial $\partial^\alpha\lm(\D)$
in~\eqref{equ:rewTheta}. Indeed, we could swap $\partial^\alpha$ and
$\lm(\D)$ to get the new rule
$\lm(\D)\partial^\alpha\rewTheta r(\D)\partial^\alpha$. This rule is
simpler in the sense that it does not require to apply any commutation 
law to its right-hand side in contrast with~\eqref{equ:rewTheta}.
However, we do not take this rule into account since it would break the
algebraic model of partial derivative equations. Indeed, if $y(x)$ is a
solution of~\eqref{equ:PDE_system}, then the relation
$(\lm(\D_i)\partial^\alpha y)(x)=(r(\D_i)\partial^\alpha y)(x)$ does not
hold in general, as illustrated in~\ref{it:ODE_rew} of the following
example.
\smallskip

\begin{example}\label{ex:diff_operators_rew}
  We continue Example~\ref{ex:diff_operators_init}.
  \begin{enumerate}
  \item\label{it:ODE_rew} Let $\Theta:=\{\D\}$ where
    $\D:=\partial-x\in\Weyl{1}$. Since $\partial$ is greater than $1$ for
    every monomial order, $\rewTheta$ is induced by the rewriting rules
    $\partial^n\rewTheta \partial^{n-1}x$, where $n$ is a strictly
    postive integer. In particular, we have the following rewriting
    sequence:
    \[\partial^2\rewTheta\partial x=x\partial+1\rewTheta x^2+1.
    \smallskip\]
    In terms of the corresponding differential equation $y'(x)=xy(x)$,
    this rewriting sequence has the following meaning. First, notice that
    the space of solutions of this equation is the one-dimensional
    $\R$-vector space spanned by the function $e^{x^2/2}$. The second
    order derivative of a solution $y(x)=Ce^{x^2/2}$, for an arbitrary
    constant $C$, is given by $y''(x)=(x^2+1)Ce^{x^2/2}$, which reads
    $(\partial^2y)(x)=(x^2+1)y(x)$ in terms of operators. Notice that if
    we allow to reduce the left $\partial$ in $\partial^2$, then we get
    $\partial^2\transTheta x^2$, which is false in terms of operators
    since $y''(x)$ is not equal to $x^2y(x)$.
  \item\label{it:Janet_example_rew} Let $\Theta:=\{\D_1,\D_2\}$, where
    $\D_1:=\partial_3^2-x_2\partial_1^2$ and $\D_2:=\partial_2^2$
    correspond to the two equations of the Janet example. We define
    $\prec$ as being the deg-lex order on
    $\Mon(\partial_1,\partial_2,\partial_3)$ induced by
    $\partial_1\prec\partial_2\prec\partial_3$, so that $\rewTheta$ is
    induced by the rewriting rules
    $\partial_3^2\rewTheta x_2\partial_1^2$ and
    $\partial_2^2\rewTheta 0$. Then, $\rewTheta$ is not confluent since:
    \begin{equation}\label{equ:non_conf_Janet_ex}
      \begin{tikzcd}
        \partial_2^2\partial_3^2\ar[d, "_\Theta"']\ar[r, "_\Theta"'] &
        \partial_2^2(x_2\partial_1^2)\ar[d, "_\Theta"]\\
        0 & 2\partial_1^2\partial_2
      \end{tikzcd}
    \end{equation}
    The right arrow is an application of the rule $\partial_2^2 \rewTheta 0$, mad
    possible by the relation
    $\partial_2^2(x_2\partial_1^2)=2 \partial_1^2\partial_2+x_2\partial_1^2\partial_2^2$, which is proven applying twice
    the commutation law $\partial_2x_2=x_2\partial_2+1$.  We deduce
    from~\eqref{equ:non_conf_Janet_ex} that any solution $y(x)$ of the equations
    $(\D_iy)(x)=0$ has to verify the new integrability condition $y_{112}(x)=0$.
  \end{enumerate}
\end{example}
\smallskip

\subsection{Involutive divisions and strategies}
\label{sec:involutive_divisions_ and_strategies}

In this section, we interpret involutive divisions in terms of strategies
for the rewriting relation induced by a set of monic differential
operators. From this, we show that the rewriting system induced by an
involutive set of operators is confluent.
\medskip

We first recall from~\cite{MR1627129} the definition of involutive
divisions and associated notions that are involutive divisors,
multiplicative variables, and autoreducibility. For that, we temporally
work with monomials instead of operators and denote these monomials with
Latin letters $u,m$ instead of~$\partial^\alpha$. Then, we will reuse the
operator notation for monomials when we will consider rewriting systems
on rational Weyl algebras. An {\em involutive division} $L$ on
$\Mon(\Delta)$ is defined by a binary relation $\divInv{L}^U$ on
$U\times\Mon(\Delta)$, for every finite subset $U\subset\Mon(\Delta)$,
satisfying for every $u,u'\in U$ and every $m,m'\in\Mon(\Delta)$, the
following relations:
\begin{enumerate}[label=\alph*)]
\item\label{it:div} $u\divInv{L}^Um\Rightarrow u\mid m$,
\item\label{it:unit} $u\divInv{L}^Uu$,
\item\label{it:mul} $u\divInv{L}^Uum$ and $u\divInv{L}^Uum'$ if and only
  if $u\divInv{L}^Uumm'$,
\item\label{it:vertex} $u\divInv{L}^Um$ and $u'\divInv{L}^Um$ implies
  $u\divInv{L}^Uu'$ or $u'\divInv{L}^Uu$,
\item\label{it:transitivity} $u\divInv{L}^Uu'$ and $u'\divInv{L}^Um$
  implies $u\divInv{L}^Um$,
\item\label{it:filter} for every $V\subseteq U$ and every $v\in V$,
  $v\divInv{L}^Um$ implies $v\divInv{L}^Vm$. 
\end{enumerate}
In the sequel, we write $\divInv{L}$ instead if $\divInv{L}^U$ when the
context is clear. We say that $u\in U$ is an {\em L-involutive divisor}
of a monomial $m$ if $u\divInv{L}m$. The variable~$\partial_i$ is said to be
{\em L-multiplicative} for $u$ w.r.t.\ $U$ if $u$ is an $L$-involutive
divisor of $\partial_iu$. Notice that $u\divInv{L}m$ if and only if
$m=m'u$, where $m'$ contains only $L$-multiplicative variables for $u$
w.r.t.\ $U$. Notice also that an involutive division is entirely
determined by the list of multiplicative variables w.r.t.\ each finite
set $U$ such that conditions \ref{it:vertex}, \ref{it:transitivity}, and
\ref{it:filter} are fulfilled. We say that $U$ is {\em L-autoreduced} if
every $u\in U$ admits only $u$ as $L$-involutive divisor, \ie,
$u'\divInv{L}u$ implies $u'=u$. Notice that if $U$ is $L$-autoreduced,
then every monomial $m$ admits at most one $L$-involutive divisor. We
finish this discussion on involutive divisions with three classical
examples. Before, let us introduce the following notation: given a
monomial $m=\partial^\alpha\in\monBasis$, let us denote by
$d_k(m):=\alpha_k$ the degree of $m$ w.r.t. the variable $\partial_k$.
\smallskip

\begin{example}\label{ex:involutive_division}

  We fix a finite set of monomials $U\subset\monBasis$. The
  {\em Janet, Thomas} and {\em Pommaret} divisions are the involutive
  divisions $\divInv{J},\divInv{T}$, and $\divInv{P}$ such that the
  variable $\partial_i$, where $1\leq i\leq n$, is
  $J,L$ or $P$-multiplicative for $u$ w.r.t.\ if and only if 
  \begin{itemize}
  \item for $\divInv{J}$: $d_i(u)=\max\{d_i(u'):\ u'\in U\ \text{and}\
    d_j(u')=d_j(u),\ \forall i<j\leq n\}$, 
  \item for $\divInv{T}$: $d_i(u)=\max\{d_i(u'):\ u'\in U\}$,
  \item for $\divInv{P}$: for every $1\leq j\leq i$, we have $d_j(u)=0$.
  \end{itemize}

  
\end{example}
\smallskip

Now, we return to differential operators and we fix a monomial order
$\prec$ on $\monBasis$. Given a finite set $\Theta\subset\Weyl{n}$ of
differential operators, all the theory of monomial sets can be applied to
the case where $U$ is the set 
\[\lm(\Theta):=\left\{\lm(\D):\ \D\in\Theta\right\}\subset\monBasis
\smallskip\]
of leading monomials of elements of $\Theta$. Hence, we may extend the
autoreducibility property for monomial sets w.r.t.\ an involutive
division to sets of differential operators.
\smallskip


\begin{definition}
  Let $\Theta\subset\Weyl{n}$ be a finite set of differential operators,
  let $\prec$ be a monomial order, and let $L$ be an involutive division
  on $\Mon(\Delta)$. We say that $\Theta$ is {\em left L-autoreduced} if
  $\lm(\Theta)$ is $L$-autoreduced.
\end{definition}
\smallskip
\noindent
The adjective "left" is here to emphasis that it may exist
$\D,\D'\in\Theta$ such that $\lm(\D)$ is an $L$-involutive divisor of a
monomial $\partial^\alpha\in\supp(r(\D'))$.
\medskip

\begin{example}\label{ex:multiplicative_variables}
  We can now apply the involutive divisions of Example
  \ref{ex:involutive_division} to find the multiplicative variables associated
  to the differential operators of Example \ref{ex:diff_operators_rew}.
  \begin{enumerate}
  \item Take $\Theta = \{\D\}$, where $\D = \partial - x \in \Weyl 1$. Then
    $\lm \D = \partial$, and $\partial$ is a multiplicative variable for $\D$ for the Janet
    and Thomas divisions, but not for the Pommaret one. This means that
    $\partial \divInv{J}^\Theta \partial^n$ and $\partial \divInv{T}^\Theta \partial^n$ for all
    $n > 0$, but that $\partial \nmid_P^\Theta \partial^n$, unless $n = 1$.

    In addition, since $\partial$ does not appear as the leading monomial of an other
    element of $\Theta$, $\D$ is left-autoreduced for all three involutive divisions.
  \item Take now $\Theta = \{\D_1,\D_2\}$, where
    $\D_1 = \partial_3^2 - x_2\partial_1^2$ and $\D_2 = \partial_2^2$. Then the following table gives
    the multiplicative variables for $\D_1$ and $\D_2$ for all three involutive divisions:
    \begin{center}
    \begin{tabular}{l|ccc}
      & Janet & Thomas & Pommaret \\ \hline
      $\D_1$ & $\partial_1, \partial_2, \partial_3$ & $\partial_1, \partial_3$ & $\emptyset$ \\
      $\D_2$ & $\partial_1, \partial_2$ & $\partial_2$ & $\partial_1$ \\
    \end{tabular}
  \end{center}

  Once again, the leading monomials of elements of $\Theta$ do not divide each
  others, so $\Theta$ is left-autoreduced for all three involutive divisions.
  \end{enumerate}
\end{example}
\smallskip


From now on, we fix a set $\Theta$ of monic (the order being fixed, we
drop it in $\prec$-monic) differential operators. Let $\RTheta$  be the
set of rewriting rules of the form
$\partial^\alpha\lm(\D)\rewTheta\partial^\alpha r(\D)$, such as
in~\eqref{equ:rewTheta}. Since $\lm(\Theta)$ is the only monomial set we
will work with, we omit it in the symbol of the involutive division: we
write $\lm(\D)\divInv{L}\partial^\alpha\lm(\D)$ when $\partial^\alpha$
contains only $L$-multiplicative variables for $\lm(\D)$ w.r.t.\
$\lm(\Theta)$. Finally, we let
\begin{equation}\label{equ:S-strategy}
  \SThetaL:=\Big\{\partial^\alpha\lm(\D)\rewThetaS{L}\partial^\alpha
  r(\D) : \, \D\in\Theta,\quad\lm(\D)\divInv{L}\partial^\alpha\lm(\D)\Big\}.
  \smallskip
\end{equation}
Here again, we choose to write
$\partial^\alpha\lm(\D)\rewThetaS{L}\partial^\alpha r(\D)$ instead of
$\partial^\alpha\lm(\D)\twoheadrightarrow_{\SThetaL}\partial^\alpha
r(\D_i)$ in order to simplify notations.
\smallskip

\begin{proposition}\label{prop:involutive_strategy}
  Let L be an involutive division on $\Mon(\Delta)$ such that $\Theta$ is
  left L-autoreduced. Then $\SThetaL$ is a strategy for $\RTheta$.  
\end{proposition}

\begin{proof}
  If the set $\Theta$ is left $L$-autoreduced, then every monomial admits
  at most one $L$-involutive divisor. Moreover, every left-hand side
  $\partial^\alpha\lm(\D)$ of a rewriting rule of $\SThetaL$ is
  $L$-divisible by~$\lm(\D)$. Hence, left-hand sides of $\SThetaL$ are
  pairwise distinct, which means that $\SThetaL$ is a pre-strategy for
  $\RTheta$. Finally, this pre-strategy is terminating since $\rewTheta$
  is terminating, so that~$\SThetaL$ is a strategy.
\end{proof}
\smallskip


From Proposition~\ref{prop:involutive_strategy}, any involutive division
$L$ such that $\Theta$ is left $L$-autoreduced induces a strategy
$\SThetaL$ for $\RTheta$. Hence, we get a well-defined normalisation
operator $\SThetaL$ corresponding to this strategy. The following
definition is an adaptation of the notion of involutive bases for
polynomial ideals~\cite{MR1627129} to the case of sets of monic
differential operators.
\smallskip

\begin{definition}
  Let $\Theta\subset\Weyl{n}$ be a finite set of differential operators,
  let $\prec$ be a monomial order on $\monBasis$ such that each element
  of $\Theta$ is monic, and let $L$ be an involutive division on
  $\Mon(\Delta)$ such that $\Theta$ is left $L$-autoreduced. We say that
  $\Theta$ is {\em $L$-involutive} if for every $\D\in\Theta$ and every
  $\partial^\alpha\in\Mon(\Delta)$, we have
  $\SThetaLNF(\partial^\alpha\D)=0$. 
\end{definition}
\smallskip

\begin{example}
Let us continue Example \ref{ex:multiplicative_variables}.
\begin{enumerate}
\item In the case $\Theta = \{ \D \}$, with $\D = \partial - x$. For the Pommaret division,
  we have seen that $\D$ admits no multiplicative variable, so the strategy
  $S_{\Theta,P}$ is reduced to the rule $\partial \rewThetaS{P} x$. As a result we get:
  \[
    \partial \D = \partial^2 - \partial x = \partial^2 - x \partial - 1 \rewThetaS{P} \partial^2 - x^2 - 1.
  \]
  This last term is a normal form for $\rewThetaS{P}$, hence
  $\SThetaNF{P}(\partial\D) \neq 0$ and so $\Theta$ is not $P$-involutive.

  
  On the other hand for the Janet and Thomas divisions, $S_{\Theta,J}$ and
  $S_{\Theta,T}$ coincide, and contain the rules
  $\partial^{n+1} \rewThetaS{J} \partial^n x$. This yields:
  \[
    \partial\D = \partial^2 - \partial x = \partial^2 - x \partial - 1 \rewThetaS{J} \partial x  - x^2 -1 = x \partial - x^2 \rewThetaS{J} 0.
  \]
  So we get $\SThetaNF{J}(\partial\D) = 0$, and more generally
  $\SThetaNF{J}(\partial^n\D) = 0$: $\Theta$ is both  $J$- and $T$-involutive.
  
\item In the case $\Theta = \{ \D_1 , \D_2 \}$, with
  $\D_1 = \partial_3^2 - x_2 \partial_1^2$ and $\D_2 = \partial_2^2$, $\Theta$ will not be involutive for
  either of the three involutive divisions of Example
  \ref{ex:involutive_division}. In the case of the Janet division for example, we have:
  \[
    \partial_3^2 \D_2 = \partial_2^2 \partial_3^2
    \rewThetaS{J} \partial_2^2 (x_2 \partial_1^2) =
    x_2 \partial_1^2 \partial_2^2 - 2 \partial_1^2 \partial_2
    \rewThetaS{J} 2 \partial_1^2 \partial_2.
  \]
  This last term is a normal form for $S_{\Theta,J}$, so we get
  $\rewThetaSNF{J}(\partial_3^2 \D_2) = 2 \partial_1^2 \partial_2 \neq 0$: $\Theta$ is not $J$-involutive. 
\end{enumerate}
\end{example}

The astute reader may remark that the last computation of the previous example
is closely related to the diagram appearing in Example
\ref{ex:diff_operators_rew}, which shows that $\rewTheta$ fails to be
confluent. This relationship between confluence and $L$-involutivity is actually
a very general one, as shown by the following theorem.

\medskip

\begin{theorem}
  Let $\Theta\subset\Weyl{n}$ be a finite set of differential operators,
  let $\prec$ be a monomial order on $\monBasis$ such that each element
  of $\Theta$ is monic, and let L be an involutive division 
  on~$\monBasis$ such that $\Theta$ is left L-autoreduced and
  L-involutive. Then, the rewriting relation~$\rewTheta$ is confluent.
\end{theorem}

\begin{proof}
  Let $\SThetaL$ be the strategy for $\RTheta$ defined such as
  in~\eqref{equ:S-strategy}. Since rewriting rules of $\RTheta$ are of
  the form $\partial^\alpha\lm(\D)\rewTheta\partial^\alpha R(\D)$, where
  $\D\in\Theta$ and $\partial^\alpha\in\monBasis$, the assumption that
  $\Theta$ is $L$-involutive means that $\rewTheta$ is
  $\SThetaL$-confluent. By Theorem~\ref{thm:S-confluence_criterion},
  $\rewTheta$ is confluent.
\end{proof}
\medskip

The end of this section aims to show that
axioms~\ref{it:div}--\ref{it:transitivity} in the definition of an
involutive division may be formulated in a purely rewriting language
using strategies. We fix a strategy~$S$ for $\rewTheta$. For every
$\D\in\Theta$, we say that $\lm(\D)$ {\em S-divides} the monomial
$\partial^\alpha\in\monBasis$ if $S$ contains a rewriting rule of the
form $\partial^\alpha\lm(\D)\rewS\partial^\alpha r(\D)$ and we say that
the variable $\partial_i\in\Delta$ is {\em S-multiplicative} for $\D$ if
$\partial_i$ is $S$-divisible by $\lm(\D)$. 
\smallskip

\begin{definition}
  A strategy $S$ for $\RTheta$ is said to be {\em involutive} if for
  every left-hand side $\partial^\alpha\lm(\D)$ of a rewriting rule in
  $S$, then $\partial^\alpha$ contains only $S$-multiplicatrice variables
  of $\D$.
\end{definition}
\smallskip

\begin{proposition}
  If the strategy $S$ is involutive, then the S-division satisfies
  axioms~\ref{it:div}--\ref{it:transitivity} of the definition of an
  involutive division. Moreover, if L is an involutive division on
  $\monBasis$ such that $\Theta$ is left L-autoreduced, then the
  $\SThetaL$-division is the restriction of L to $\lm(\Theta)$.
\end{proposition}

\begin{proof}
  Let us show the first assertion. Axioms~\ref{it:div}, \ref{it:vertex},
  and~\ref{it:transitivity} hold since $S$ is a strategy for $\RTheta$.
  Indeed, left-hand sides of $\RTheta$ are of the form
  $\partial^\alpha\lm(\D)$, hence~\ref{it:div}, and left-hand sides of
  elements of $S$ are pairwise distinct, hence~\ref{it:vertex}
  and~\ref{it:transitivity}. Moreover, axioms~\ref{it:unit}
  and~\ref{it:mul} hold by definition of an involutive strategy.

  Let us show the second assertion. By definition of the strategy
  $\SThetaL$ and of the $\SThetaL$-division, $\lm(\D)$ has the same set
  of multiplicative variables for $L$ and for the $\SThetaL$-division.
  Hence, a monomial $\partial^\alpha$ is $L$-divisible by $\lm(\D)$, with
  $\D\in\Theta$, if and only if it is $\SThetaL$-divisible by $\lm(\D)$.
  That proves the assertion.
\end{proof}

\paragraph{Conclusion.} We introduced a sufficient condition,
based on deterministic reduction strategies, of confluence for
rewriting systems on vector spaces. As a particular case,
we recover the diamond lemma. This work maybe extended in particular
into two main directions. The first one consists in weakening our assumption
on the set $\K$ of coefficients, by allowing non invertible coefficients.
A second extension consists in characterising Janet bases in this framework,
with the objective to develop constructive methods in the analysis and formal
resolution of PDE's.

\bibliography{Biblio}

\end{document}

%%% Local Variables:
%%% mode: latex
%%% TeX-master: t
%%% End:
