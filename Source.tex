\documentclass[10pt]{easychair}
\usepackage[T1]{fontenc}
\usepackage[utf8]{inputenc}
\usepackage[english]{babel}
\usepackage{amsmath}
\usepackage{amssymb,mathrsfs}
\pagestyle{plain}
\usepackage{enumitem} 
\usepackage{amsthm}
\usepackage{titlesec}
\usepackage[all]{xy}
\usepackage{algorithm}
\usepackage{algorithmic}

\newtheorem{theorem}{Theorem}[section]
\newtheorem{lemma}[theorem]{Lemma}
\newtheorem{proposition}[theorem]{Proposition}
\newtheorem{corollary}[theorem]{Corollary}

\theoremstyle{definition}
\newtheorem{definition}[theorem]{Definition}
\newtheorem{property}[theorem]{Property}
\newtheorem{remark}[theorem]{Remark}
\newtheorem{example}[theorem]{Example}


\newcommand\K{\mathbb{K}}
\newcommand\KX{\K X}
\newcommand\supp{\text{supp}}
\newcommand\rewR{\underset{R}{\longrightarrow}}
\newcommand\rewh{\underset{h}{\longrightarrow}}
\newcommand\transR{\overset{*}{\rewR}}
\newcommand\transh{\overset{*}{\rewh}}
\newcommand\equivR{\underset{R}{\overset{*}{\longleftrightarrow}}}
\newcommand\equivh{\underset{h}{\overset{*}{\longleftrightarrow}}}


\newcommand\todo[1]{\textcolor{red}{#1.}}

\begin{document}

\title{Title}

\author{
Cyrille Chenavier\inst{1}
\and
Maxime Lucas\inst{2}
}

\institute{
  Inria Lille - Nord Europe, équipe Valse\\
  \email{cyrille.chenavier@inria.fr}
\and
Inria Rennes - Bretagne Atlantique, équipe Gallinette\\
\email{maxime.lucas@inria.fr}
}

\authorrunning{Chenavier and Lucas}
\titlerunning{Running title}

\maketitle

\begin{abstract}

  
\end{abstract}
 
\section{Introduction}

\section{Well-formed rewriting steps}

We fix a commutative field $\K$ as well as a well-founded ordered set
$(X,<)$. We denote by $\KX$ the vector space spanned by~$X$: an element
$v\in\KX$ is a finite formal linear combination of elements of $X$ with
coefficients in $\K$. In particular, for every $v\in\KX$, there exists a
unique finite set $\supp(v)\subset X$, called the \emph{support} of $v$,
such that
\begin{equation}\label{equ:support}
  v=\sum_{x\in\supp(v)}\lambda_xx\ \text{and}\ x\in\supp(v)\Rightarrow\lambda_x
  \neq 0.
\end{equation}

\smallskip
\noindent
We denote by $\supp(v)^c=X\setminus\supp(v)$. The sum of
$u=\sum\lambda_xx$ and $v=\sum\mu_xx$ equals $\sum(\lambda_x+\mu_x)x$ and
the product of $\lambda\in\K$ by $v$ equals $\sum(\lambda\lambda_x)x$. We
extend the order $<$ into the multiset order, still written $<$, on
$\KX$: we have $u<v$ if for every $x\in\supp(u)\cap\supp(v)^c$, there
exists $y\in\supp(v)\cap\supp(u)^c$ such that $y>x$.

\medskip

We fix a set $R\subseteq X\times\KX$ which represents rewrite rules of
the form $x\rewR r$. The set $R$ induces the rewriting relation on $\KX$,
still written $\rewR$, defined as follows:
\begin{equation}\label{equ:rewriting_step}
  \sum\lambda_xx+v\rewR\sum\lambda_xr_x+v,
\end{equation}
whenever $\lambda_x\neq 0$, $x\rewR r_x\in R$ and $x\notin\supp(v)$. 

\begin{definition}
A \emph{local strategy} for $R$ is the choice, for every $x\in X$ not minimal for $<$, of a rewriting rule $h_x=x \rewR r_x$ such that $r_x < x$.
\end{definition}

Suppose chosen such a local strategy $h$. 
Any vector $v$ can be decomposed in a unique way as
$\sum\lambda_xx+v'$, where $y\in\supp(v')$ implies that $y$ is minimal
for $<$, and $x\in\supp(v)\cap\supp(v')^c$ is not. We define a rewriting
relation $\rewh$ as follows:
\begin{equation}\label{equ:well-formed_rewriting_step}
  \sum\lambda_xx+v'\rewh\sum\lambda_xr_x+v',
\end{equation}
where for every $x$, $h_x=x \rewR r_x$.

\medskip

\begin{definition}
  A vector $v$ is said to be a \emph{$h$-normal form} if it is a normal
  form for $\rewh$.
\end{definition}


\begin{example}\label{ex:h_norma_form}
Let $X=\{x,y,z,t\}$, $x\rewR y$, $y \rewR z + t$, $z \rewR y - t$. Note that this is not terminating since we have the infinite loop $y \rewR z + t \rewR (y - t) + t = y$. We choose the order $x > y > z$, and the following distinguished rewrite rules: $h_x = x \rewh y$, $h_y = y \rewh z + t$. Then the $R$-normal forms are the $\lambda_t t$, while the $h$-normal form are all the $\lambda_t t + \lambda_z z$.
\end{example}


\begin{lemma}\label{lem:h_normal_forms}
  Let $v$ be a vector in $\KX$. Either $v$ is minimal for $<$, or there exists $v'<v$ such that $v\rewh v'$. In particular, $h$-normal forms are
  precisely the minimal elements of $\KX$ for $<$. 
\end{lemma}

\medskip

For each $v\in\KX$, there exists at most one $v'$ such that $v \rewh v'$,
and $\rewh$ is compatible with the termination order $<$. As a
consequence, any $v \in \KX$ is sent by multiple applications of $\rewh$
to a unique $h$-normal form that we denote by $H(v)$. This defines a map
$H : \KX \to \KX$.  

\begin{proposition}\label{prop:linearity_of_H}
  The map $H$ is a linear projector.
\end{proposition}

\begin{proof}
  The $h$-normal forms are closed under sums, so that $H(H(v))=H(v)$ for
  every $v$, that is $H$ is a projector. Moreover, if $u \rewh u'$ and
  $v \rewh v'$, then we have $u + v \rewh u' + v'$. Iterating~$\rewh$, we
  get $H(u + v) = H (H(u)+H(v))=H(u)+H(v)$. 
\end{proof}

\section{A confluence criterion}




\begin{definition}\label{def:standardisation_property}
  We say that $R$ is $h$-confluent if for every rewrite rule $x\rewR v\in R$, we have $\ x-v\in\ker(H)$.
\end{definition}

\begin{example}
Let us take the same example as in \ref{ex:h_norma_form}. Then $H(x) = H(y) = z + t$, with $H(z) = z = H(y - t)$, and so $R$ is $h$-confluent. Replacing the rule $z \rewR y - t$ by $z \rewR y$, we get $H(z) = z$ and $H(y) = z + t$, so $R$ is not $h$-confluent anymore. 
\end{example}

\begin{proposition}\label{prop:equationnal_theory}
  If $R$ is $h$-confluent, then
  $u \equivR v$ if and only if $u-v\in\ker(H)$.
\end{proposition}

\begin{proof}
The relation $\equivR$ is the closure of $\rewR$ under transitivity, symmetry and sum. Since the relation $u - v \in \ker(H)$ is closed under these operations, we get one implication.

Reciprocally, if $u - v \in \ker(H)$ then by definition of $H$ we have $u \equivh v$, and in particular $u \equivR v$.
\end{proof}

\begin{proposition}
If $R$ is $h$-confluent, then $\rewR$ is confluent.
\end{proposition}
\begin{proof}
  Let $v,\ v_1,\ v_2\in\KX$ be such that $v\transR v_i$, for $i=1,\ 2$.
  From Proposition~\ref{prop:equationnal_theory}, $v_1-v_2$ belongs to
  $\ker(H)$, that is $H(v_1)=H(v_2)$. Hence, we get
  \[
    \xymatrix @C = 4em @R = 1.5em{
      &
      v_1
      \ar@/^/ [rd] ^{*}
      & \\
      v
      \ar@/^/ [ru] ^{*}
      \ar@/_/ [rd] _{*}
      &
      &
      H(v_i)
      \\
      &
      v_2
      \ar@/_/  [ru] _{*}
      &
    }
  \]
  
\end{proof}

In Theorem~\ref{thm:confluence_criterion}, we introduce a confluence
criterion when $R$ satisfies~\ref{def:standardisation_property}. For
that, we assume that $R$ is equipped with a well-founded order $\prec$
satisfying the following decreasingness property:

\begin{definition}\label{proper:decreasingness_property}
  We say that $R$ is \emph{locally $h$-confluent} if for every non
  minimal $x\in X$ and $r=x\rewR v$, then letting $h_x=x\rewh r_x$, we
  have the confluence diagram:
  
  \[
    \xymatrix @C = 4em @R = 1.5em{
      &
      r_x
      \ar@{.>}@/^/ [rd] ^{*}
      & \\
      x
      \ar@/^/ [ru] ^{h_x}
      \ar@/_/ [rd] _{r}
      &
      &
      v'
      \\
      &
      v
      \ar@{.>}@/_/  [ru] _{*}
      &
    }
  \]
  where each rewriting step occurring in the dotted arrows are strictly
  smaller than $r$ for $\prec$.
\end{definition}

\begin{theorem}\label{thm:confluence_criterion}
  If $R$ is locally $h$-confluent, then $R$ is $h$-confluent. In particular, $\rewR$ is confluent.
\end{theorem}

\begin{proof}
 We reason by induction on $r$. Looking at the square corresponding to $r$:
    \[
    \xymatrix @C = 4em @R = 1.5em{
      &
      r_x
      \ar@{.>}@/^/ [rd] ^{*}
      & \\
      x
      \ar@/^/ [ru] ^{h_x}
      \ar@/_/ [rd] _{r}
      &
      &
      v'
      \\
      &
      v
      \ar@{.>}@/_/  [ru] _{*}
      &
    }
  \]
  We have $H(x) = H(r_x)$ by definition of $H$, and $H(r_x) = H(v') = H(v)$ by induction hypothesis, which concludes the proof.
\end{proof}

\begin{remark}
Let us show how Newman's Lemma fits as a particular case of our set up. 
Suppose that $R$ is locally confluent and terminating. We define the relation $x > y$ on $X$ as the transitive closure of the relation $x \rewR u$ and $y$ appears in $u$: this is a well-founded relation. By definition if $x \in X$ is not minimal for $>$ then $x$ is not an $R$-normal form. Let us fix an arbitrary rewriting step $h_x= x \rewh r_x$.  By definition of $>$, $r_x < x$ so that $h$ is a local strategy. Ordering the rewrite rules by their sources makes $R$ locally $h$-confluent, and Theorem \ref{thm:confluence_criterion} finally shows that $R$ is confluent.  
\end{remark}

\end{document}

