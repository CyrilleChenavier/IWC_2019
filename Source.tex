\documentclass[11pt]{article}
\usepackage[utf8]{inputenc}
\usepackage[english]{babel}
\usepackage{amsmath,amssymb,mathrsfs,amsthm}
\pagestyle{plain}
\usepackage[filecolor=green]{hyperref}
\usepackage{color}
\usepackage{tikz}
\usepackage{tikz-cd}
\usepackage{enumitem}
\usepackage{vmargin}
\setmarginsrb{2cm}{2cm}{2cm}{1.5cm}{0cm}{0cm}{0cm}{1.5cm}
\bibliographystyle{plain}

% Newtheorems
\newtheorem{theorem}{Theorem}[section]
\newtheorem{lemma}[theorem]{Lemma}
\newtheorem{proposition}[theorem]{Proposition}
\newtheorem{corollary}[theorem]{Corollary}

\theoremstyle{definition}
\newtheorem{definition}[theorem]{Definition}
\newtheorem{remark}[theorem]{Remark}
\newtheorem{example}[theorem]{Example}

% Newcommand

% Texte
\newcommand\eg{\emph{e.g.}}
\newcommand\ie{\emph{i.e.}}
\newcommand\wrt{w.r.t.}
\newcommand\G{Gröbner}
\newcommand\todo[1]{{\bf\textcolor{red}{#1}}}

% Raccourcis mathématiques
\newcommand\diff[1]{\partial_{#1}}
\newcommand\D{\mathcal{D}}
\newcommand\basis{\mathscr{B}}
\newcommand\SigmaTheta{\Sigma_{\Theta}}
\newcommand\ordS{\leq_S}
\newcommand\ordR{\leq_R}
\DeclareMathOperator{\supp}{supp}
\DeclareMathOperator{\lm}{lm}
\DeclareMathOperator{\lc}{lc}
\DeclareMathOperator{\id}{id}

% Structures algébriques
\newcommand\K{\mathbb{K}}
\newcommand\Q{\mathbb{Q}}
\newcommand\R{\mathbb{R}} 
\newcommand\N{\mathbb{N}}
\newcommand\QX{\mathbb{Q}[x_1,\cdots,x_n]}
\newcommand\QXX{\mathbb{Q}(x_1,\cdots,x_n)}
\newcommand\Weyl[1]{B_{#1}(\Q)}
\newcommand\monBasis{\Mon(\Delta)}
\newcommand\Span[1]{\langle #1\rangle}
\DeclareMathOperator{\Mon}{Mon}
\DeclareMathOperator{\im}{im}

% Réécriture dans les ev
\newcommand\rewR{\to_R}
\newcommand\parRew{\twoheadrightarrow}
\newcommand\parS{\twoheadrightarrow_S}
\newcommand\parTheta[1]{\twoheadrightarrow_{\Theta,#1}}

\newcommand\transRew{\overset{*}{\to}}
\newcommand\transR{\overset{*}{\to}_R}
\newcommand\transS{\overset{*}{\twoheadrightarrow}_S}

\newcommand\rewEquiv{\overset{*}{\leftrightarrow}}
\newcommand\equivR{\overset{*}{\leftrightarrow}_R}

\DeclareMathOperator{\SNF}{{\it S}-NF}

% Réécriture dans les algèbres de Weyl
\newcommand\rewTheta{\to_\Theta}
\newcommand\transTheta{\overset{*}{\to}_\Theta}
\newcommand\divInv[1]{\mid_{#1}}
\newcommand\RTheta{R_{\Theta}}
\newcommand\SThetaL{S_{\Theta,L}}
\newcommand{\SThetaNF}[1]{{\it S}_{\Theta,#1}\operatorname{-NF}}
\DeclareMathOperator{\SThetaLNF}{{\it S}_{\Theta,L}-NF}

\begin{document}

\title{Strategies for linear rewriting systems:\\[0.2cm]
 link with parallel rewriting and involutive divisions\vspace{0.5cm}}
\author{Cyrille Chenavier\footnote{
    Johannes Kepler University, Institute for Algebra,
    cyrille.chenavier@jku.at.
  }\and Maxime Lucas\footnote{Inria Rennes - Bretagne Atlantique,
    Gallinette team, maxime.lucas@inria.fr.}
  }
\date{}

\maketitle
      
\begin{abstract}
  We study rewriting systems whose underlying set of terms is equipped with a
  vector space structure over a given field. We introduce parallel rewriting
  relations, which are rewriting relations compatible with the vector space
  structure, as well as rewriting strategies, which consist in choosing one
  rewriting step for each reducible basis element of the vector space. Using
  these notions, we introduce the $S$-confluence property and show that it
  implies confluence. We deduce a proof of the diamond's lemma, based on
  strategies. We illustrate our general framework with rewriting systems over
  rational Weyl algebras, that are vector spaces over a field of rational
  functions. In particular, we show that involutive divisions induce rewriting
  strategies over rational Weyl algebras, and using the $S$-confluence property,
  we show that involutive sets induce confluent rewriting systems over rational
  Weyl algebras.
\end{abstract}
\noindent
\begin{small}\textbf{Keywords:} confluence, parallel rewriting, rewriting
  strategies, involutive divisions.\\[0.2cm]
  \textbf{M.S.C 2010 - Primary:} 13N10, 68Q42. \textbf{Secondary:} 12H05,
  35A25.
\end{small}

\tableofcontents

\section{Introduction}

Rewriting systems are computational models given by a set of syntactic
expressions and transformation rules used to simplify expressions into
equivalent ones. Since rewriting theory is applicable to different
problems of mathematics and computer science, it was developed for many
syntaxes of terms, \eg, strings, ($\Sigma-$, higher-order, infinitary)
terms, graphs, (commutative, noncommutative, vectors of) polynomials,
(linear combinations of) trees, (higher-dimensional) cells. Abstract
rewriting theory unifies these contexts and provides universal
formulations of rewriting properties, such as termination, normalisation
and (local) confluence. Newman's lemma is one of the most famous results 
of abstract rewriting and asserts that under termination hypothesis, 
local confluence implies confluence.
\medskip

In the context of rewriting over algebraic structures, the Newman's lemma
is used in conjunction with the critical pairs lemma to algorithmically
prove confluence. This is something fundamental since confluent rewriting
systems provide methods for solving decision problems, computing (linear,
homotopy) bases, Hilbert series, or free
resolutions~\cite{MR846601, GuiraudHoffbeckMalbos19, MR2964639,
  MR1072284, MR1360005}. From these methods, one get constructive proofs
of theoretical results, such as  embedding, coherence or homological
theorems~\cite{MR506890, MR0506423, MR3347996, MR3742562, MR265437,
  MR920522}, but also applications to problems coming from topics
modelled by algebra, such as cryptography, analysis of (ordinary
differential, partial derivative, time-delay) equations or control
theory. For instance, many informations of functional equations may be
read over free resolutions: integrability conditions, parametrization of
solutions, existence of autonomous curves~\cite{MR2233761, MR1308976}.
\medskip

When one considers algebraic structures with underlying vector space
operations, the conjunction of Newman's lemma and the critical pairs
lemma is traditionally known under the name of the diamond's lemma. In
practice, this lemma is used to test if a generating set of a polynomial
ideal is a \G\ basis, since confluent linear rewriting systems are
usually induced by \G\ bases or one of their numerous adaptations to
different classes of algebras or operads~\cite{MR506890, MR2202562,
  MR2667136, MR1044911, MR1299371}. As an illustration of theses classes,
let us mention polynomial Weyl algebras that are models of differential
operators with polynomial coefficients. These algebras are composed of
polynomials over two sets of $n$ variables, the state variables
$x_1,\ldots,x_n$ and the vector field variables
$\partial_1,\ldots,\partial_n$, and submitted to the commutation rules
\[ \forall 1\leq i\neq j\leq n: \qquad x_ix_j=x_jx_i,
\quad \partial_i\partial_j=\partial_j\partial_i, \quad \partial_ix_j=
x_j\partial_i, \quad\partial_ix_i=x_i\partial_i+1. \]
These relations represent classical rules from differential calculus: the
second one means that second order derivatives of smooth functions
commute, the third one means that $x_j$ is constant for differentiation
with respect to $x_i$ and the last one represents the Leibniz's rule for
differentiation with respect to $x_i$, that is,
$\partial_i(x_if)=x_i\partial_i(f)+f$, for any smooth function $f$.
Rewriting over vector spaces requires to introduce a notion of
well-formed rewriting step, also called {\em positive rewriting step}
in~\cite{GuiraudHoffbeckMalbos19}, that is, each step reduces only one
basis element together with its coefficient in a given vector. Typically,
the vector is a polynomial and the reduced basis element is a monomial
that appears in this polynomial with a nonzero coefficient. Doing so
avoids pathological situations, \eg, $v_1\to v_2$ implies that the
rewriting step $v_2=v_2-v_1+v_1\to v_2-v_2+v_1=v_1$ is not well-formed.
On the other hand, well-formed rewriting steps are not compatible with
vector space operations since as soon as two different basis elements are
rewritten in the well-formed rewriting steps $u_1\to u_2$ and
$v_1\to v_2$, then $u_1+v_1\to u_2+v_2$ is not well-formed.
\medskip

The notion of well-formed rewriting step is specific to rewriting systems over
vector spaces and, as mentioned above, it is not compatible with the underlying
algebraic operations. This lack of compatibility makes the theory of linear
rewriting rather painful, for instance the proof of the critical pair lemma is
more involved than for string rewriting (see~\cite[Theorem
4.2.1]{GuiraudHoffbeckMalbos19}). In our view, these observations call for the
development of a theory of linear rewriting free from well-formed rewriting
steps. In the long run, we hope that this will contribute to bridge the gap
between abstract and linear rewriting, for instance to find a common proof of
Newman's lemma and of the diamond lemma.

\begin{center}
  {\large\bf Our results}
\end{center}

In the present paper, we introduce an alternative approach to rewriting theory
over vector spaces, which does not use the notion of well-formed rewriting
step. Instead, our rewriting steps only depend on vector spaces operations, and they
may reduce many basis elements at once, still avoiding pathological
situations. Moreover, our framework is valid in every vector space and can be
applied to the case where the coefficients do not commute with basis
elements. We illustrate this last point with rewriting systems over rational
Weyl algebras, and from this, we show that so-called {\em involutive
  bases}~\cite{MR1627129} induce confluent rewriting systems.  \medskip

Given a set of rewriting rules of the form $e\to v$, where $e$ is a basis element
and $v$ is a vector, we introduce the notion of {\em parallel rewriting
  relation} with rewriting steps $v_1\parRew v_2$ that consists in replacing
each left-hand side of a rule $e$ occurring in $v_1$ by the corresponding
right-hand side $v$ to get~$v_2$. As mentioned above, this definition is purely
internal to the category of vector spaces and does not require any notion of
well-formed rewriting step. Our approach is related to the classical one through
the notion of {\em strategy}, which means that the rewriting preorder induced by
$\parRew$ is terminating. Following the ideas of~\cite{GuiraudHoffbeckMalbos19},
and contrary to rewriting with \G\ bases, rewriting with strategies does not
require any monomial order but uses the order induced by the rewriting process
itself. The link with classical rewriting over vector spaces is given by a
confluence criterion. Indeed, the initial rewriting rules induce a rewriting
relation~$\rewR$ involving well-formed rewriting steps only, and we say that
$\rewR$ is {\em S-confluent} if for every rewriting rule $e\to v$, $e$ and~$v$ are
joinable using $\parRew$. In Theorem~\ref{thm:S-confluence_criterion}, we show
that the $S$-confluence property implies confluence of $\rewR$, and we construct
a basis of the vector space quotiented by the equivalence relation of $\rewR$ in
terms of normal forms for $\parRew$. Moreover, we also show that the
$S$-confluence property is characterised by a decreasingness property, which
enables us to provide a new proof of the diamond's lemma in
Theorem~\ref{thm:diamond_lemma}.  \medskip

Since our approach works for arbitrary vector spaces, it may be declined
in different classes of algebras over fields, including rational Weyl
algebras. The later extend polynomial Weyl algebras presented above,
since they are composed of differential operators with coefficients in
the field of rational functions. Unlike polynomial Weyl algebras, that
are vector spaces over the field of constants and modules over the ring
of polynomials, rational Weyl algebras are vector spaces over the field
of rational functions. Notice that because of the Leibniz's rule,
rational functions do not commute with operators. Rewriting-like methods
in this context yield applications to formal analysis of linear systems
of ordinary differential or partial derivative equations as mentioned
above. In particular, involutive divisions, such as {\em Janet, Thomas}
and {\em Pommaret} divisions, provide deterministic techniques to rewrite
differential operators. By determinism, we mean that each differential
operator admits at most one involutive divisor, that is, it may be
rewritten into at most one other differential operator. This determinism
is strongly related to our notion of strategy and parallel reductions. In
particular, we show in  Theorem~\ref{thm:involutive_conf} that involutive
bases induce $S$-confluent rewriting systems. Finally, we show how most
of the axioms of involutive divisions may be formalised in purely
rewriting language using rewriting strategies.
\medskip

\paragraph{Organisation.}

In Section~\ref{sec:parallel_rewriting_relations_over_vector_space}, we
present our general framework and general results for rewriting systems
over vector spaces. In
Section~\ref{sec:rewriting_strategies_over_vector_space}, we introduce
parallel rewriting relations, rewriting strategies, and a normalisation
operator associated to each strategy. In
Section~\ref{sec:confluence_relative_to_a_strategy}, we introduce the
$S$-confluence property, and show that it implies confluence property for
well-formed rewriting relations and that bases of quotient vector spaces
may be constructed in terms of normal forms for parallel rewriting
relations. We also show that $S$-confluence is characterised by a
decreasingness property, from which we deduce a proof of the diamond's
lemma based on strategies. In
Section~\ref{sec:rewriting_strategies_over_rational_Weyl_algebras}, we
illustrate our general framework by rewriting systems over rational 
Weyl algebras. In Section~\ref{sec:rewriting_systems_over_Weyl_algebras},
we introduce well-formed rewriting systems over rational Weyl algebras. 
In Section~\ref{sec:involutive_divisions_and_strategies}, we recall the
definition of an involutive division and of involutive sets of operators
and show that involutive divisions define rewriting strategies, and that
involutive sets $S$-confluent rewriting relations.
\medskip

\paragraph{Terminology and conventions.}

Throughout the paper, we use the standard terminology and conventions of
rewriting theory~\cite{MR1629216}. An {\em abstract rewriting system} is
a pair $(A,\to)$, where $A$ is a set and $\to$ is binary relation on $A$,
called {\em rewriting relation}. An element~$(a,b)\in\to$ is written
$a\to b$ and is called a {\em rewriting step}. A {\em normal form} for
$\to$ is an element $a\in A$ such that there is no $b\in A$ such that
$a\to b$. We denote by $\transRew$  (respectively,~$\rewEquiv$) the
closure of $\to$ under transitivity and reflexivity (respectively, and
symmetry). The equivalence class of $a\in A$ modulo the equivalence
relation $\rewEquiv$ is written $[a]_{\rewEquiv}$ and the set of all
equivalence classes is written $A/\rewEquiv$. When $a\transRew b$, that
is, there exists a (possibly empty) finite sequence of rewriting steps
from $a$ to  $b$, we say that $a$ {\em rewrites} into $b$. The rewriting
relation $\to$ is said to be {\em confluent} if whenever $a\transRew b$
and $a\transRew c$, then $b$ and $c$ are {\em joinable}, that is, there
exists $d$ such that $b\transRew c$ and $c\transRew d$. The confluence
property is equivalent to the {\em Church-Rosser property} that assterts
that whenever $a\rewEquiv b$, then~$a$ and $b$ are joinable.
\medskip

We also recall the notion of support of a vector in a given basis of a
vector space. Given a vector space $V$ over the ground field $\K$ and a
$\basis$ of $V$, every vector $v$ admits a unique finite decomposition
with respect to the basis $\basis$ and coefficients in the ground field:

\begin{equation}\label{equ:vector_decompo}
  v=\sum\lambda_ie_i,\quad\lambda_i\neq 0.
\end{equation}
The set of basis elements which appear in the decomposition
\eqref{equ:vector_decompo} is called the {\it support} of $v$ and is
written $\supp(v)$. 


\section{Rewriting strategies over vector spaces}
\label{sec:rewriting_strategies_over_vector_space}

\todo{Presentation section}
\medskip

Throughout the section, we fix a ground field $\K$, a $\K$-vector space
$V$, and a basis $\basis$ of $V$. We say vectors and basis elements for
elements of $V$ and $\basis$, respectively. 

\subsection{Confluence relative to a strategy}
\label{sec:confluence_relative_to_a_strategy}

\todo{Presentation intro, motiver stratégies et fusionner avec le prochain
parapgraphe}
\medskip

We fix a subset $R$ of $\basis\times V$. In Section \todo{ref}, the set
$R$ represents a set of rewriting rules over~$V$, and for this reason, we
denote its elements by $e\rewR v$ instead of $(e,v)$ and call them
rewriting rules. Let us denote by $\Span{R}$ the subspace of $V$ spanned
by elements $e-v$, where $e\rewR v$ is a rewriting rule. The quotient
vector space of $V$ modulo $\Span{R}$ is written $V/\Span{R}$. 
\medskip

We fix a subset~$S$ of~$R$ with pairwise distinct left-hand sides, that
is, given two rules $e\rewR v$ and $e'\rewR v'$ in $S$ such that $e=e'$,
we have $e\rewR v=e'\rewR v'$. Let us consider the endomorphism
$r_S:V\to V$ defined by $r_S(e)=v$ if there exists a rule $e\rewR v$ in
$S$ with left-hand side $e$, and $r_S(e)=e$ if there is no rule in $S$
with left-hand side $e$. We consider the rewriting relation $\parS$ on
$V$ defined by $u\parS u'$ whenever $u'=r_S(u)$. In the sequel, we refer
this relation as being the parallel rewriting relation induced by $S$;
the terminology parallel makes explicit that all basis vectors in
$\supp(u)$ are reduced at once with rules in $S$ and justifies the double
head arrows in $\parS$. Notice that the parallel rewriting relation is
stable under vector spaces operations, that is, for every rewriting steps
$u\parS u'$, $v\parS v'$, and for every scalar $\lambda\in\K$, there
exists a rewriting step $u+\lambda v\parS u'+\lambda v'$. Moreover, also
notice that $\parS$ is deterministic in the sense that for every vector
$u$, there is at most one~$u'$ such that $u\parS u'$. Let us denote by
$\ordS$ the rewriting preorder on~$\basis$ induced by~$\parS$, that is,
$\ordS$ is the transitive closure of the relation: there exists a vector
$u$ such that $e'\transS u$ and $e\in\supp(u)$. In other words,
$e\ordS e'$ if and only if $e\in\supp(r_S^n(e'))$, where $n$ is a
nonnegative integer and $r_S^n$ is the $n$-th composition of $r_S$. 
\medskip

\begin{definition}\label{def:strategies}
  A {\em prestrategy} for $R$ is a subset $S$ of $R$ with pairwise
  distinct left-hand sides. A {\em strategy} for $R$ is a prestrategy
  such that the rewriting preorder $\ordS$ of $\parS$ is Noetherian. 
\end{definition}
\smallskip

Let $S$ be a strategy for $R$. In the sequel, we denote elements of $S$
in the form $e\parS r_S(e)$. The rewriting preorder $\ordS$ being
Noetherian, every vector $u$ admits an {\em S-normal form}, that is
another vector $u'$ such that $u\transS u'$ and $r_S$ acts trivially 
on $u'$. Indeed, $u'$ is the limit of the stationnary sequence
$(r_S^n(u))_n$. Moreover, since $\parS$ is determinitic, this $S$-normal
form is unique and we denote it by~$\SNF(u)$. 
\smallskip

\begin{example}\label{ex:strategies_step_1}
  Let us illustrate (pre)strategies with the following
  (counter-)examples.
  \begin{enumerate}
  \item\label{it:ex_strat_1} Assume that $V$ is $4$-dimensional with
    basis $\basis=\{e_1,e_2,e_3,e_4\}$ and let us consider the following
    set of rewriting rules:
    \[R=\{e_1\rewR e_2,\quad e_2\rewR e_3+e_4,\quad e_3\rewR e_2-e_4\}.\]
    Considering the prestrategy
    \[S=\{e_1\rewR r_S(e_1)=e_2,\quad e_2\rewR r_S(e_2)=e_3+e_4\}\subset
    R,\]
    the preorder $\ordS$ is Noetherian since we have
    $e_1>_Se_2>_Se_3,e_4$, and $e_3,e_4$ are minimal. Hence,~$S$ is a
    strategy for $R$. The unique $S$-normal form of
    $u=\lambda_1e_1+\cdots+\lambda_4e_4$ is computed as follows:
    \[u\parS\lambda_1e_2+(\lambda_2+\lambda_3)e_3+
    (\lambda_2+\lambda_4)e_4\parS(\lambda_1+\lambda_2+\lambda_3)e_3+
    (\lambda_1+\lambda_2+\lambda_4)e_4,\]
    which yields:
    \[\SNF(u)=(\lambda_1+\lambda_2+\lambda_3)e_3+
    (\lambda_1+\lambda_2+\lambda_4)e_4.\]
  \item\label{it:c-ex_strat_1} Assume that $V$ is $3$-dimensional with
    basis $\basis=\{e_1,e_2,e_3\}$, let us consider the set of rewriting
    rules:
    \[R=\{e_1\rewR e_2+e_3,\quad e_2\rewR e_1,\quad e_3\rewR -e_1\},\]
    and $S=R$. Then, $S$ is not a strategy since the preorder $\ordS$ is
    not Noetherian. In fact,~$\ordS$ is cyclic since from the rewriting
    sequence $e_2\parS e_1\parS e_2+e_3$, we get $e_2>_Se_3$, so that
    $e_1>_S e_2>_S e_3>_S e_1>_S\cdots$. Notice however that each vector
    $u=\lambda_1e_1+\lambda_2e_2+\lambda_3e_3$ admits a unique $S$-normal
    form which is $0$:
    \[u=\lambda_1e_1+\lambda_2e_2+\lambda_3e_3\parS(\lambda_2-\lambda_3)
    e_1+\lambda_1e_2+\lambda_1e_3\parS(\lambda_2-\lambda_3)e_2+(\lambda_2-
    \lambda_3)e_3\parS 0.
    \]
  \item\label{it:case_N} Let $V$ be the vector space with basis
    $\basis= \mathbb N$ and consider the set of rewriting rules:
    \[R=\{n\rewR n+1:\quad n\in\N\}.\]
    Then, a prestrategy $S$ corresponds to a subset $E$ of $\mathbb N$.
    Moreover, such a prestrategy is a strategy if and only if for all
    $n\in E$, there exists $k\in \mathbb N$ such that $n + k \notin E$. 
  \end{enumerate}
\end{example}
\smallskip

In the next proposition, we establish properties of the map
$\SNF:V\to V,\ v\mapsto\SNF(v)$ that we use in \todo{compélter, et dire
  qu'à partir de maintenant S est une stratégie}.
\smallskip

\begin{proposition}\label{prop:SNF_projector}
  The map $\SNF$ is a linear projector.
\end{proposition}

\begin{proof}
  For every vector $u\in V$, we have $r_S(\SNF(u))=\SNF(u)$, which proves 
  $\SNF\circ\SNF=\SNF$. Moreover, given another vector $v\in V$, let
  $k\in\N$ be an integer such that $\SNF(u)=r_S^k(u)$ and
  $\SNF(v)=r_S^k(v)$. For every scalar $\lambda\in\K$, we have 
  \[r^k_S(u+\lambda v)=r^k_S(u)+\lambda r^k_S(v)=\SNF(u)+\lambda\SNF(v).
  \]
  Hence, $\SNF(u+\lambda v)=r^k_S(u+\lambda v)$ is equal to
  $\SNF(u)+\lambda\SNF(v)$, which proves that $\SNF$ is linear.
\end{proof}

\smallskip

Now, we introduce the $S$-confluence property. 
\smallskip

\begin{definition}\label{def:standardisation_property}
  Given a strategy $S$ for $R$, we say that $R$ is \emph{S-confluent} if
  for every rewriting rule $e\rewR v$ in $R$, we have $\SNF(e-v)=0$.
  \todo{rajouter diagramme}
\end{definition}
\smallskip

The following theorem illustrates how $S$-confluence provides a method
for constructing a basis of the vector space $V/\Span{R}$. 
\medskip

\begin{theorem}\label{thm:S-confluence_criterion}
  Let $R$ be a set of rewriting rules and let $S$ be a strategy for $R$.
  If $R$ is $S$-confluent, then we have a vector spaces isomorphism
  \begin{equation}\label{equ:iso_SNF}
    V/\Span{R}\quad\simeq\quad\im(\SNF).
  \end{equation}
  In particular, $\{e+\Span{R}:\ \SNF(e)=e\}$ is a basis of
  $V/\Span{R}$.
\end{theorem}

\begin{proof}
  For proving~\eqref{equ:iso_SNF}, it is sufficient to show that
  $\ker(\SNF)$ is equal to $\Span{R}$. Let us show the first inclusion.
  From Proposition~\ref{prop:SNF_projector}, $\SNF$ is a projector, so
  that its kernel is equal to the image of the operator $\id_V-\SNF$.
  Moreover, by definition of the map $r_S$, for every vector $u\in V$, we
  have $u-r_S(u)\in\Span{R}$, which gives $u-\SNF(u)\in\Span{R}$ by
  induction. Hence, $\ker(\SNF)=\im(\id_V-\SNF)$ is included in
  $\Span{R}$. Let us show the converse inclusion. From the $S$-confluence
  hypothesis, for every rewriting rule $e\rewR v$, we have $\SNF(e-v)=0$. 
  Since all elements $e-v$, for $e\rewR v$, generate $\Span{R}$, we 
  deduce that the latter is included in $\ker(\SNF)$, which concludes the
  proof of~\eqref{equ:iso_SNF}. The second assertion of the theorem is a
  consequence of the fact that $\im(\SNF)$ has a basis composed by basis
  elements that are fixed by $\SNF$.
\end{proof}
\smallskip

\begin{example}\label{ex:S-conf}
  Let us continue Point~\ref{it:ex_strat_1} of
  Example~\ref{ex:strategies_step_1}. The following identities hold:
  \[\begin{split}
  \SNF(e_1)=e_3+e_4=\SNF(e_2),&\quad\SNF(e_2)=e_3+e_4=\SNF(e_3+e_4)
  \\[0.3cm]
  \SNF(e_3)=& \ e_3=\SNF(e_2-e_4),
  \end{split}
  \smallskip\]
  so that $R$ is $S$-confluent. Notice that if we replace the rule
  $e_3\rewR e_2-e_4$ in $R$ by $e_3\rewR e_2$, we get $\SNF(e_3)=e_3$ and
  $\SNF(e_2)=e_3+e_4$, so $\rewR$ is not $S$-confluent anymore. 
\end{example}

\subsection{Strategies for rewriting relations}

In this section, we relate parallel rewriting to the traditional approach
of rewriting relations over vector spaces that consist in reducing one
basis element at each step. In particular, we show that the
$S$-confluence property implies confluence for traditional rewriting
relations, and we give a new proof of the diamond's lemma, based on
$S$-confluence.
\medskip

As previously, we fix a set $R\subset\basis\times V$ of rewriting rules,
whose elements are written $e\rewR v$. Moreover, we also impose that for
every such rule, we have $e\notin\supp(v)$. These rules are extended into
a rewriting relation on $V$, still written $\rewR$, with rewriting steps
of the following form:  
\begin{equation}\label{equ:R_rewriting_step}
  \lambda e+u\rewR\lambda v+u,
  \smallskip
\end{equation}
where $e\rewR v\in R$ is a rewriting rule, $\lambda$ is a nonzero scalar
and $u$ is a vector such that $e$ does not belong to $\supp(u)$. A normal
form for $\rewR$ is called an {\em R-normal form}. In the sequel, we
refer~$\rewR$ as the rewriting relation induced by the rewriting rules
$R$. The relation~$\rewR$ is not stable under vector spaces
operations, since $u_1\rewR u_2$, $v_1\rewR v_2$, and $\mu\in\K$
generally do not imply $\mu u_1+v_1\rewR \mu u_2+v_2$. In contrast, 
Proposition~\ref{prop:vs_structure} shows that $\equivR$ is compatible
with these operations. In the proof of this proposition, we use the
following lemma.
\smallskip

\begin{lemma}\label{lem:butterfly}
  If we have $u_1\equivR u_2$ and $v_1\equivR v_2$, then,
  $\mu u_1+v_1\equivR\mu u_2+v_2$ holds for every $\mu\in\K$.
\end{lemma}

\begin{proof}
  The proof is an adaptation
  of~\cite[Lemma 3.1.3]{GuiraudHoffbeckMalbos19}. We first show the
  following particular case:
  \begin{equation}\label{equ:sum_rew}
    u_1\rewR u_2\quad\Rightarrow\quad\mu u_1+v_1\equivR\mu u_2+v_1.
  \end{equation}
  By
  definition of $\rewR$, we have $u_1=\lambda e+u$ and $u_2=\lambda v+u$,
  where $e\rewR v$ is a rewriting rule,~$\lambda$ is a scalar and $e$
  does not belong to $\supp(u)$. Let $\nu$ be the coefficient of $e$ in
  $v_1$, so that we may write $v_1=\nu e+v_1'$, where $e$ does not belong
  to $\supp(v_1')$. Since $\mu u_1+v_1=(\mu\lambda+\nu)e+\mu u+v_1'$,
  $\mu u_2+v_1=\nu e+\mu\lambda v +\mu u+v_1'$, and, $e\notin\supp(v)$,
  we have $\mu u_1+v_1\transR(\mu\lambda+\nu)v+\mu u+v_1'
  \overset{*}{\leftarrow}\mu u_2+v_1$, which proves~\eqref{equ:sum_rew}.
  If $u_1\equivR u_2$, using~\eqref{equ:sum_rew}, an induction on the
  length of the path $u_1\equivR u_2$ shows that
  $\mu u_1+v_1\equivR\mu u_2+v_1$, and by an analogous argument, we have
  $\mu u_2+v_1\equivR\mu u_2+v_2$. Hence, we have
  $\mu u_1+v_1\equivR\mu u_2+v_2$, which concludes the proof.
\end{proof}
\smallskip

\begin{proposition}\label{prop:vs_structure}
  Given two vectors $u,u'\in V$, we have $u\equivR u'$ if and only if
  $u=u'+\Span{R}$.
\end{proposition}

\begin{proof}
  By definition of the rewriting relation $\rewR$, if $u\rewR u'$, then
  $u=u'+\Span{R}$. Since $\equivR$ is the smallest equivalence relation
  that contains $\rewR$, we deduce that $u\equivR u'$ also implies
  $u=u'+\Span{R}$. Conversely, let us write $u-u'=\sum\mu(e-v)$, where
  $\mu$ are scalars and $e\rewR v$ are rewriting rules. Since
  $e\equivR v$ holds for each term of the sum, Lemma~\ref{lem:butterfly}
  implies that $u\equivR u'$, which concludes the proof.
\end{proof}
\smallskip

As a consequence of the previous proposition, $V/\equivR$ admits a vector
space structure, isomorphic to $V/\Span{R}$, for the operations
\[[u]_{\equivR}+\lambda[v]_{\equivR}=[u+\lambda v]_{\equivR}.\smallskip\]
\noindent
Notice that Proposition~\ref{prop:vs_structure} is not true if we do not
assume that $e\notin\supp(v)$ holds for every rewriting rule~$e\rewR v$: 
for instance, if we have only one rule $e\rewR 2e$, then $e\in\Span{R}$ 
but $[e]_{\equivR}\neq[0]_{\equivR}$. We point out that the assumption
$e\notin\supp(v)$ was explicitly used in the proof of
Lemma~\ref{lem:butterfly}.
\medskip

Now, let us relate strategies for $R$ to the rewriting relation $\rewR$.
\smallskip

\begin{lemma}\label{lem:strategies}
  Let $S$ be a strategy for $R$. Then, the following inclusion holds:
  \[\parS\quad\subset\quad\transR.\]
\end{lemma}

\begin{proof}
  Since $S$ is a strategy, the rewriting preorder $\ordS$ of $\parS$ is
  Noetherian. This preorder induces the following Noetherian order, still
  written $\ordS$, on $V$: we let $u'\ordS u$ if $\supp(u')$ is smaller
  than $\supp(u)$ for the multiset order of $\ordS$. We show the
  proposition by induction along $\ordS$. If $u$ is minimal, then
  $r_S(u)=u$ and hence $u\transR r_S(u)$. Suppose now that $u$ is not
  minimal. Then $u$ may be uniquely written in the form
  \begin{equation}\label{equ:decompo_max}
    u=\sum_{i=1}^n\lambda_ie_i+u'
  \end{equation}
  where the basis elements $e_i$ are the elements of $\supp(u)$ that are
  maximal for~$\ordS$, and the $\lambda_i$'s are their coefficients in
  $u$. In particular, we have $u'<_S u$ and by induction, we have
  $u'\transR r_S(u')$. Moreover, by definition of $\ordS$, the rewriting
  rules that are involved in this rewriting sequence have left-hand sides
  strictly smaller than $e_i$'s, so that
  $u\transR\sum\lambda_ie_i+r_S(u')$. Moreover, since the $e_i$'s are not
  comparable for~$\ordS$, for each indices $i$ and $j$, $e_i$ does not
  belong to $\supp(r_S(e_j))$. Hence, we may reduce successively each
  $e_i$ into $r_S(e_i)$ and finally have
  \[
  u \quad\transR\quad
  \sum \lambda_ie_i+r_S(u')\quad\transR\quad
  \sum\lambda_ir_S(e_i) + r_S(u')
  =r_S(u).\]
\end{proof}
\smallskip

Now, we can show that the $S$-confluence property implies confluence of
$\rewR$.
\medskip

\begin{theorem}\label{thm-S-conf}
  Let $S$ be a strategy for $R$. If $R$ is $S$-confluent, then the
  rewriting relation $\rewR$ is confluent.
\end{theorem}

\begin{proof}
  It is sufficient to show that $\rewR$ has the Church-Rosser property,
  which can be done using our previous results. Let $u,u'\in V$ be two
  vectors such that $u\equivR u'$. From
  Proposition~\ref{prop:vs_structure}, we have $u=u'+\Span{R}$, from
  Theorem~\ref{thm:S-confluence_criterion}, we have $\SNF(u)=\SNF(u')$,
  and from Proposition~\ref{lem:strategies} we have $u\transR\SNF(u)$ and
  $u'\transR\SNF(u')$. All together, $u$ and $u'$ rewrite into
  $\SNF(u)=\SNF(u')$, so that~$\rewR$ has the Church-Rosser property.
\end{proof}
\smallskip

Note that $S$-confluence is a sufficient but not a necessary condition for
confluence. Indeed, with~$\basis$ the set of integers and the rewriting rules
$n\rewR n+1$ as in Point~\ref{it:case_N} of
Example~\ref{ex:strategies_step_1}, there is no strategy such that
$\rewR$ is confluent relative to this strategy.
\smallskip

\begin{example}\label{ex:conf}
  In Example~\ref{ex:S-conf}, we have shown that the following set of
  rewriting rules
  \[R=\{e_1\rewR e_2,\quad e_2\rewR e_3+e_4,\quad e_3\rewR e_2-e_4\}\]
  is $S$-confluent for the strategy defined in Point~\ref{it:ex_strat_1}
  of Example~\ref{ex:strategies_step_1}. Hence, $\rewR$ is confluent.
\end{example}
\smallskip

We finish this section by showing how the diamond's lemma fits as a
particular case of our setup. In our proof, we use the following
observation: if $\ordR$ is the rewriting preorder of $\rewR$ and if $S$
is a prestrategy for $R$, then $e\ordS e'$ implies $e\ordR e'$, for every
basis vectors $e$ and $e'$. In particular, if $\ordR$ is Noetherian, then
every prestrategy is a strategy for $R$.
\medskip

\begin{theorem}[\cite{MR506890}]\label{thm:diamond_lemma}
  Let $R$ be a set of rewriting rules such that the rewriting preorder
  $\ordR$ of~$\rewR$ is Noetherian. Assume that  for every $e\in\basis$
  such that $e\rewR v$ and $e\rewR v'$, $v$ and $v'$ are joinable. 
  Then,~$\rewR$ is confluent.
\end{theorem}

\begin{proof}
  From Theorem~\ref{thm-S-conf}, it is sufficient to show that $R$ is
  $S$-confluent for a strategy $S$ for $R$. For every basis element $e$
  that is reducible by $\rewR$, we select exactly one arbitrary rewriting
  rule with left hand-side $e$. Then, let $S$ be the prestrategy composed
  of these selected rewriting rules. Following the discussion preceding
  the statement of the theorem, $S$ is a strategy for $R$. Then, we only
  have to show that for every rule $e\rewR v$, we have $\SNF(e-v)=0$. For
  that, we construct the following diagram:
  \[\begin{tikzcd}
  e\ar[d, "_R"']\ar[r, twoheadrightarrow, "_S"'] & 
  r_S(e)\ar[r, "*", "_R"'] & \SNF(e)\ar[d, equal]\\
  v\ar[r, "*", "_R"'] & \SNF(v)\ar[r, equal] & u
  \end{tikzcd}\]
  By definition of $\SNF$, we have $r_S(e)\transS\SNF(r_S(e))=\SNF(e)$
  and $v\transS\SNF(v)$, and from Lemma~\ref{lem:strategies}, we have
  $\parS\subset\transR$. That gives the top right and bottom left arrows
  of the diagram. Moreover, from the assumption of the theorem, $\SNF(v)$
  and $\SNF(e)$ have to be joinable, so that there exists $u\in V$ such
  that $\SNF(v)$ and $\SNF(e)$ rewrite into $u$. But both $\SNF(v)$ and
  $\SNF(e)$ are normal forms for $\parS$, so that they are also normal
  forms for $\rewR$ by our choice of $S$. That gives the two equality
  symbols in the diagram. Hence, we have shown that for our strategy $S$,
  we have $\SNF(e)=\SNF(v)$, or equivalently $\SNF(e-v)=0$.
\end{proof}

\section{Rewriting strategies over rational Weyl algebras}
\label{sec:rewriting_strategies_over_rational_Weyl_algebras}

In this section, we investigate rewriting systems over rational Weyl
algebras and relate involutive divisions to rewriting strategies for such
systems. In particular, we show that involutive sets in rational Weyl
algebras induce confluent rewriting systems.
\medskip

Throughout the section, we fix a set $X=\{x_1,\cdots,x_n\}$ of
indeterminates and the field of fractions of the polynomial algebra $\QX$
over~$\Q$ is denoted by $\Q(X):=\QXX$, it is the set of rational
functions in the indeterminates $X$. We fix another set of
variables $\Delta=\{\diff{1},\cdots,\diff{n}\}$ that model partial
derivative operators, see Example~\ref{ex:diff_operators_init}. We denote
by $\partial^{\alpha}:=\diff{1}^{\alpha_1}\cdots\diff{n}^{\alpha_n}$ the
monomial over $\Delta$ with multi-exponent
$\alpha=(\alpha_1,\cdots,\alpha_n)\in\N^n$. Finally, let $\monBasis$ be
the set of monomials over~$\Delta$:
\[\monBasis:=\left\{\partial^\alpha:\ \alpha\in\N^n\right\}.
\smallskip\]
In what follows, we keep the terminology monomials only for elements of
$\Mon(\Delta)$ and not for elements in $\Mon(X)$.

\subsection{Rewriting systems over rational Weyl algebras}
\label{sec:rewriting_systems_over_Weyl_algebras}

In this section, we recall the definition of the rational Weyl algebra
and introduce rewriting systems on the rational Weyl algebra induced by
monic operators.
\medskip

\begin{definition}
  The {\it rational Weyl algebra} over $\Q(X)$ is the set of polynomials
  $\Q(X)[\Delta]$ with coefficients in $\Q(X)$ and indeterminates
  $\Delta$. The multiplication of this $\mathbb Q$-algebra is induced by
  the commutation laws $\partial_i\partial_j=\partial_j\partial_i$ and
  \[\diff{i}f=f\diff{i}+\frac{d}{dx_i}(f),\quad f\in\Q(X),\quad
  1\leq i\leq n,
  \smallskip\]
  where $d/dx_i:\Q(X)\to\Q(X)$ is the partial derivative operator with
  respect to~$x_i$. This algebra is denoted by $\Weyl{n}$.
\end{definition}
\smallskip

Notice that $\Weyl{n}$ is a $\Q(X)$-vector space and that the monomial 
set $\monBasis$ is a basis of~$\Weyl{n}$. Elements of~$\Weyl{n}$ should 
be thought of as differential operators whose coefficients are rational
functions, and for this reason, a generic element of this algebra is
denoted by $\D$ and is called a differential operator. In the following
example, we illustrate how these operators provide an algebraic model of
linear systems of ordinary differential (in the case $n=1$) and partial
derivative equations (in the case~$n\geq 2$) with one unknown function. 
\smallskip

\begin{example}\label{ex:diff_operators_init}
  {\color{white}toto}
  \begin{enumerate}
  \item\label{it:ODE_init} The linear ordinary differential equation
    $y'(x)=xy(x)$ is written in the form $(\D y)(x)=0$, where the operator
    $\D:=\partial-x$ belongs to $\Weyl{1}=\Q(x)[\partial]$. 
  \item\label{it:Janet_example_init} Consider Janet's
    example~\cite{MR1308976}, that is, the linear system of partial
    derivative equations with~$3$ variables, one unknown function, and
    the two equations $y_{33}(x)=x_2y_{11}(x)$ and $y_{22}(x)=0$, where
    $y_{ij}(x)$ denotes the second order derivative of the unknown
    function $y(x)$ with respect to the variables $x_i$ and~$x_j$. Then,
    these equations are written $(\D_1y)(x)=0$ and $(\D_2y)(x)=0$, where
    $\D_1,\D_2\in\Weyl{3}$ are defined as follows:
    \[\D_1:=\partial_3^2-x_2\partial_1^2,\quad \D_2:=\partial_2^2.
    \smallskip\]
  \end{enumerate}
\end{example}

\begin{remark}
  In~\ref{it:ODE_init} of Example~\ref{ex:diff_operators_init}, we
  implicitly used that every $f\in\Q(X)$ induces a unique multiplication
  operator $y(x)\mapsto f(x)y(x)$.
\end{remark}
\smallskip

The next step before introducing rewriting systems over rational Weyl
algebras is to recall the definition of monic operators. We fix a
monomial order $\prec$ on $\monBasis$, that is, a terminating total 
order which is admissible, \ie, $\partial^{\alpha}\prec\partial^{\beta}$
implies $\partial^{\alpha+\gamma}\prec\partial^{\beta+\gamma}$, for every
$\alpha,\beta,\gamma\in\N^n$. Given an operator $\D$, we denote by
$\lm(\D)$ the leading monomial of $\D$ with respect to~$\prec$, that 
is, $\lm(\D)$ is the greatest element of $\supp(\D)$, where the support 
is defined w.r.t.\ the basis $\monBasis$. 
\smallskip

\begin{definition}
  Let $\prec$ be a monomial order on $\monBasis$. A differential
  operator $\D\in\Weyl{n}$ is said to be $\prec$-{\em monic} if the
  coefficient of $\lm(\D)$ on $\D$ is equal to $1$. Moreover, given a
  monic differential operator $\D$, we denote by $r(\D):=\lm(\D)-\D$.
\end{definition}
\smallskip

Since the monomial order $\prec$ is fixed, me simply say monic instead of
$\prec$-monic. Given a set of monic operators~$\Theta\subseteq\Weyl{n}$,
let us consider the rewriting relation on $\Weyl{n}$ induced by the
following rewriting rules: 
\begin{equation}\label{equ:rewTheta}
  \RTheta:=\Big\{\partial^\alpha\lm(\D)\to_{\RTheta}\partial^\alpha
  r(\D):\ \D\in\Theta,\ \partial^\alpha\in\Mon(\Delta)\Big\}.
\end{equation}
For simplicity, we write $\D\rewTheta\D'$ instead of
$\D\to_{R_\Theta}\D'$. The rewriting relation $\rewTheta$ is terminating
since the rewriting rules reduce a monomial into a combination of
strictly smaller monomials w.r.t.\ the terminating order $\prec$.
Moreover, notice that in the case where the coefficient $\lc(\D)\in\Q(X)$
of $\lm(\D)$ in $\D$ is not constant, the situation is much harder.
Indeed, in this case, the left-hand sides of the rewriting rules are of
the form $\partial^\alpha(\lc(\D)\lm(\D))$ and due to commutation laws,
these elements are not monomials. In particular, we are not in the
situation of our general approach developed in  
Section~\ref{sec:parallel_rewriting_relations_over_vector_space} anymore.
\medskip

We finish this section with some comments on $\rewTheta$. Let us consider 
the linear system of ordinary differential or partial derivative
equations with unknown function $y$ given by 
\begin{equation}\label{equ:PDE_system}
  \{(\D y)=0:\D\in\Theta\}.
\end{equation}
Let $y(x)$ be an arbitrary solution to this system. Then, for every
operator $\partial^\alpha$ and every $\D\in\Theta$, we also have
$(\partial^\alpha\D y)(x)=0$, or equivalently,
$(\partial^\alpha\lm(\D)y)(x)=(\partial^\alpha r(\D)y)(x)$. Hence, if
there is a rewriting path $\D_1\transTheta\D_2$, then the solution $y(x)$
of~\eqref{equ:PDE_system} satisfies $(\D_1y)(x)=(\D_2y)(x)$. This remark
has deep applications in the formal theory of partial differential
equations, for instance for finding integrability conditions or computing
dimensions of solution spaces, see~\cite{MR1308976}. Moreover, notice
that since $\monBasis$ is a commutative set, there is another possible
choice for rewriting the monomial $\partial^\alpha\lm(\D)$
in~\eqref{equ:rewTheta}. Indeed, we could swap $\partial^\alpha$ and
$\lm(\D)$ to get the new rule
$\lm(\D)\partial^\alpha\rewTheta r(\D)\partial^\alpha$. This rule is
simpler in the sense that it does not require to apply any commutation 
law to its right-hand side in contrast with~\eqref{equ:rewTheta}.
However, we do not take this rule into account since it would break the
algebraic model of partial derivative equations. Indeed, if $y(x)$ is a
solution of~\eqref{equ:PDE_system}, then the relation
$(\lm(\D_i)\partial^\alpha y)(x)=(r(\D_i)\partial^\alpha y)(x)$ does not
hold in general, as illustrated in~\ref{it:ODE_rew} of the following
example.
\smallskip

\begin{example}\label{ex:diff_operators_rew}
  We continue Example~\ref{ex:diff_operators_init}.
  \begin{enumerate}
  \item\label{it:ODE_rew} Let $\Theta:=\{\D\}$ where
    $\D:=\partial-x\in\Weyl{1}$. Since $\partial$ is greater than $1$ for
    every monomial order, $\rewTheta$ is induced by the rewriting rules
    $\partial^n\rewTheta \partial^{n-1}x$, where $n$ is a strictly
    positive integer. In particular, we have the following rewriting
    sequence:
    \[\partial^2\rewTheta\partial x=x\partial+1\rewTheta x^2+1.
    \smallskip\]
    In terms of the corresponding differential equation $y'(x)=xy(x)$,
    this rewriting sequence has the following meaning. First, notice that
    the space of solutions of this equation is the one-dimensional
    $\R$-vector space spanned by the function $e^{x^2/2}$. Moreover, the
    second order derivative of a solution $y(x)=Ce^{x^2/2}$, for an
    arbitrary constant $C$, is given by the formula
    $y''(x)=(x^2+1)Ce^{x^2/2}$, 
    which reads $(\partial^2y)(x)=(x^2+1)y(x)$ in terms of operators.
    Notice that if we allow to reduce the left~$\partial$ 
    in~$\partial^2$, then we get $\partial^2\transTheta x^2$, which is 
    false in terms of operators since $y''(x)$ is not equal to~$x^2y(x)$.
  \item\label{it:Janet_example_rew} Let $\Theta:=\{\D_1,\D_2\}$, where
    $\D_1:=\partial_3^2-x_2\partial_1^2$ and $\D_2:=\partial_2^2$
    correspond to the two equations of the Janet example. We define
    $\prec$ as being the deg-lex order on
    $\Mon(\partial_1,\partial_2,\partial_3)$ induced by
    $\partial_1\prec\partial_2\prec\partial_3$, so that $\rewTheta$ is
    induced by the rewriting rules
    $\partial_3^2\rewTheta x_2\partial_1^2$ and
    $\partial_2^2\rewTheta 0$. Then,~$\rewTheta$ is not confluent since:
    \begin{equation}\label{equ:non_conf_Janet_ex}
      \begin{tikzcd}
        \partial_2^2\partial_3^2\ar[d, "_\Theta"']\ar[r, "_\Theta"'] &
        \partial_2^2(x_2\partial_1^2)\ar[d, "_\Theta"]\\
        0 & 2\partial_1^2\partial_2
      \end{tikzcd}
    \end{equation}
    The right arrow is an application of the rule
    $\partial_2^2 \rewTheta 0$, made possible since
    $\partial_2^2(x_2\partial_1^2)$ is equal to
    $\partial_1^2\partial_2+x_2\partial_1^2\partial_2^2$ (to see this, it
    suffices to apply twice the commutation law
    $\partial_2x_2=x_2\partial_2+1$).  We deduce
    from~\eqref{equ:non_conf_Janet_ex} that any solution $y(x)$ of the
    equations $(\D_iy)(x)=0$ has to verify the new integrability
    condition $y_{112}(x)=0$.
  \end{enumerate}
\end{example}
\smallskip

\subsection{Involutive divisions and strategies}
\label{sec:involutive_divisions_and_strategies}

In this section, we interpret involutive divisions in terms of strategies
for the rewriting relation induced by a set of monic differential
operators. From this, we show that the rewriting system induced by an
involutive set of operators is confluent.
\medskip

We first recall from~\cite{MR1627129} the definition of involutive
divisions and associated notions that are involutive divisors,
multiplicative variables, and autoreducibility. For that, we temporally
work with monomials instead of operators and denote these monomials with
Latin letters $u,m$ instead of~$\partial^\alpha$. Then, we will reuse the
operator notation for monomials when we will consider rewriting systems
over rational Weyl algebras. An {\em involutive division} $L$ on
$\Mon(\Delta)$ is defined by a binary relation~$\divInv{L}^U$ on
$U\times\Mon(\Delta)$, for every finite subset $U\subset\Mon(\Delta)$,
satisfying for every $u,u'\in U$ and every $m,m'\in\Mon(\Delta)$, the
following relations:
\begin{enumerate}[label=\alph*)]
\item\label{it:div} $u\divInv{L}^Um\Rightarrow u\mid m$,
\item\label{it:unit} $u\divInv{L}^Uu$,
\item\label{it:mul} $u\divInv{L}^Uum$ and $u\divInv{L}^Uum'$ if and only
  if $u\divInv{L}^Uumm'$,
\item\label{it:vertex} $u\divInv{L}^Um$ and $u'\divInv{L}^Um$ implies
  $u\divInv{L}^Uu'$ or $u'\divInv{L}^Uu$,
\item\label{it:transitivity} $u\divInv{L}^Uu'$ and $u'\divInv{L}^Um$
  implies $u\divInv{L}^Um$,
\item\label{it:filter} for every $V\subseteq U$ and every $v\in V$,
  $v\divInv{L}^Um$ implies $v\divInv{L}^Vm$. 
\end{enumerate}
In the sequel, we write $\divInv{L}$ instead if $\divInv{L}^U$ when the
context is clear. We say that $u\in U$ is an {\em L-involutive divisor}
of a monomial $m$ if $u\divInv{L}m$. The variable~$\partial_i$ is said to be
{\em L-multiplicative} for $u$ w.r.t.\ $U$ if $u$ is an $L$-involutive
divisor of $\partial_iu$. Notice that $u\divInv{L}m$ if and only if
$m=m'u$, where $m'$ contains only $L$-multiplicative variables for $u$
w.r.t.\ $U$. Notice also that an involutive division is entirely
determined by the list of multiplicative variables w.r.t.\ each finite
set $U$ such that conditions \ref{it:vertex}, \ref{it:transitivity}, and
\ref{it:filter} are fulfilled. We say that $U$ is {\em L-autoreduced} if
every $u\in U$ admits only $u$ as $L$-involutive divisor, \ie,
$u'\divInv{L}u$ implies $u'=u$. Notice that if $U$ is $L$-autoreduced,
then every monomial $m$ admits at most one $L$-involutive divisor. We
finish this discussion on involutive divisions with three classical
examples. Before, let us introduce the following notation: given a
monomial $m=\partial^\alpha\in\monBasis$, let us denote by
$d_k(m):=\alpha_k$ the degree of $m$ w.r.t. the variable $\partial_k$.
\smallskip

\begin{example}\label{ex:involutive_division}

  We fix a finite set of monomials $U\subset\monBasis$. The
  {\em Janet, Thomas} and {\em Pommaret} divisions are the involutive
  divisions $\divInv{J},\divInv{T}$, and $\divInv{P}$ such that the
  variable $\partial_i$, where $1\leq i\leq n$, is
  $J,L$ or $P$-multiplicative for $u$ w.r.t.\ $U$ if and only if 
  \begin{itemize}
  \item for $\divInv{J}$: $d_i(u)=\max\{d_i(u'):\ u'\in U\ \text{and}\
    d_j(u')=d_j(u),\ \forall i<j\leq n\}$, 
  \item for $\divInv{T}$: $d_i(u)=\max\{d_i(u'):\ u'\in U\}$,
  \item for $\divInv{P}$: for every $1\leq j\leq i$, we have $d_j(u)=0$.
  \end{itemize}

  
\end{example}
\smallskip

Now, we return to differential operators and we fix a monomial order
$\prec$ on $\monBasis$. Given a finite set $\Theta\subset\Weyl{n}$ of
$\prec$-monic differential operators, all the theory of monomial sets can
be applied to the case where $U$ is the set of leading monomials of
elements of $\Theta$:
\[\lm(\Theta):=\left\{\lm(\D):\ \D\in\Theta\right\}\subset\monBasis
\smallskip\]
Hence, we may extend the autoreducibility property for monomial sets
w.r.t.\ an involutive division to sets of differential operators.
\smallskip

\begin{definition}
  Let $\Theta\subset\Weyl{n}$ be a finite set of $\prec$-monic
  differential operators, let $\prec$ be a monomial order, and let $L$ be
  an involutive division on $\Mon(\Delta)$. We say that $\Theta$ is
  {\em left L-autoreduced} if $\lm(\Theta)$ is $L$-autoreduced.
\end{definition}
\smallskip
\noindent
The adjective "left" is here to emphasis that it may exist
$\D,\D'\in\Theta$ such that $\lm(\D)$ is an $L$-involutive divisor of a
monomial $\partial^\alpha\in\supp(r(\D'))$.
\medskip

\begin{example}\label{ex:multiplicative_variables}
  We can now apply the involutive divisions of 
  Example~\ref{ex:involutive_division} to find the multiplicative
  variables associated to the differential operators of
  Example~\ref{ex:diff_operators_rew}.
  \begin{enumerate}
  \item Take $\Theta = \{\D\}$, where $\D = \partial - x \in \Weyl 1$.
    Then, $\lm(\D) =\partial$, and $\partial$ is a multiplicative
    variable for $\D$ for the Janet and Thomas divisions, but not for the
    Pommaret one. This means that $\partial \divInv{J}^\Theta \partial^n$
    and $\partial \divInv{T}^\Theta \partial^n$ for all $n > 0$, but that
    $\partial \nmid_P^\Theta \partial^n$, unless $n = 1$. In addition,
    since $\Theta$ is a singleton, it is trivially left-autoreduced for
    all three involutive divisions. 
  \item Take now $\Theta = \{\D_1,\D_2\}$, where
    $\D_1=\partial_3^2 - x_2\partial_1^2$ and $\D_2 = \partial_2^2$. The
    following table gives the multiplicative variables for $\D_1$ and
    $\D_2$ w.r.t.\ $\Theta$ for all three involutive divisions:
    \begin{center}
    \begin{tabular}{l|ccc}
      & Janet & Thomas & Pommaret \\ \hline
      $\D_1$ & $\partial_1, \partial_2, \partial_3$ & $\partial_1, \partial_3$ & $\emptyset$ \\
      $\D_2$ & $\partial_1, \partial_2$ & $\partial_2$ & $\partial_1$ \\
    \end{tabular}
  \end{center}
    Once again, the leading monomials of elements of $\Theta$ do not divide
    each others, so $\Theta$ is left-autoreduced for all three involutive
    divisions.
  \end{enumerate}
\end{example}
\smallskip


From now on, we fix a set $\Theta$ of monic (the order being fixed, we
drop it in $\prec$-monic) differential operators. Let $\RTheta$  be the
set of rewriting rules of the form
$\partial^\alpha\lm(\D)\rewTheta\partial^\alpha r(\D)$, such as
in~\eqref{equ:rewTheta}. Since $\lm(\Theta)$ is the only monomial set we
will work with, we omit it in the symbol of the involutive division: we
write $\lm(\D)\divInv{L}\partial^\alpha\lm(\D)$ when $\partial^\alpha$
contains only $L$-multiplicative variables for $\lm(\D)$ w.r.t.\
$\lm(\Theta)$. Finally, we let
\begin{equation}\label{equ:S-strategy}
  \SThetaL:=\Big\{\partial^\alpha\lm(\D)\parTheta{L}\partial^\alpha
  r(\D) : \, \D\in\Theta,\quad\lm(\D)\divInv{L}\partial^\alpha\lm(\D)
  \Big\}.
  \smallskip
\end{equation}
Here again, we choose to write
$\partial^\alpha\lm(\D)\parTheta{L}\partial^\alpha r(\D)$ instead of
$\partial^\alpha\lm(\D)\twoheadrightarrow_{\SThetaL}\partial^\alpha
r(\D_i)$ in order to simplify notations.
\smallskip

\begin{proposition}\label{prop:involutive_strategy}
  Let L be an involutive division on $\Mon(\Delta)$ such that $\Theta$ is
  left L-autoreduced. Then $\SThetaL$ is a strategy for $\RTheta$.  
\end{proposition}

\begin{proof}
  If the set $\Theta$ is left $L$-autoreduced, then every monomial admits
  at most one $L$-involutive divisor. Moreover, every left-hand side
  $\partial^\alpha\lm(\D)$ of a rewriting rule of $\SThetaL$ is
  $L$-divisible by~$\lm(\D)$. Hence, left-hand sides of $\SThetaL$ are
  pairwise distinct, which means that $\SThetaL$ is a pre-strategy for
  $\RTheta$. Finally, if $<_\Theta$ denotes the rewriting preorder of
  $\rewTheta$, then $\partial^\alpha<_\Theta\partial^\beta$ implies that
  $\partial^\alpha\prec\partial^\beta$, so that $<_\Theta$ is
  terminating. Hence, the rewriting preorder of $\twoheadrightarrow_{\SThetaL}$ is also
  terminating, and $\SThetaL$ is a strategy for~$\RTheta$.
\end{proof}
\smallskip


From Proposition~\ref{prop:involutive_strategy}, any involutive division
$L$ such that $\Theta$ is left $L$-autoreduced induces a strategy
$\SThetaL$ for $\RTheta$. Hence, we get a well-defined normalisation
operator $\SThetaLNF$ corresponding to this strategy. The following
definition is an adaptation of the notion of involutive bases for
polynomial ideals~\cite{MR1627129} to the case of sets of monic
differential operators.
\smallskip

\begin{definition}
  Let $\Theta\subset\Weyl{n}$ be a finite set of differential operators,
  let $\prec$ be a monomial order on $\monBasis$ such that each element
  of $\Theta$ is monic, and let $L$ be an involutive division on
  $\Mon(\Delta)$ such that $\Theta$ is left $L$-autoreduced. We say that
  $\Theta$ is an {\em $L$-involutive set} if for every $\D\in\Theta$ and every
  $\partial^\alpha\in\Mon(\Delta)$, we have
  $\SThetaLNF(\partial^\alpha\D)=0$. 
\end{definition}
\smallskip

\begin{example}
Let us continue Example~\ref{ex:multiplicative_variables}.
\begin{enumerate}
\item In the case $\Theta = \{ \D \}$, with $\D = \partial - x$. For the
  Pommaret division, we have seen that $\D$ admits no multiplicative
  variable, so the strategy $S_{\Theta,P}$ is reduced to the rule
  $\partial\parTheta{P} x$. As a result we get:
  \[\partial \D = \partial^2 - \partial x = \partial^2 - x \partial - 1
  \parTheta{P} \partial^2 - x^2 - 1.
  \]
  This last term is a normal form for $\parTheta{P}$, hence
  $\SThetaNF{P}(\partial\D) \neq 0$ and so $\Theta$ is not
  $P$-involutive. On the other hand for the Janet and Thomas divisions,
  $S_{\Theta,J}$ and $S_{\Theta,T}$ coincide, and contain the rules
  $\partial^{n+1}\parTheta{L}\partial^n x$, where $L=J,T$. This yields:
  \[
  \partial\D=\partial^2-\partial x=\partial^2-x\partial-1\parTheta{L}
  \partial x-x^2-1=x\partial -x^2 \parTheta{L}0.
  \]
  So we get $\SThetaNF{L}(\partial\D) = 0$, and more generally
  $\SThetaNF{L}(\partial^n\D) = 0$: $\Theta$ is both  $J$- and
  $T$-involutive.  
\item In the case $\Theta = \{ \D_1 , \D_2 \}$, with
  $\D_1 = \partial_3^2 - x_2 \partial_1^2$ and $\D_2 = \partial_2^2$,
  $\Theta$ will not be involutive for either of the three involutive
  divisions of Example~\ref{ex:involutive_division}. In the case of the
  Janet division for example, we have:
  \[
    \partial_3^2 \D_2 = \partial_2^2 \partial_3^2
    \parTheta{J} \partial_2^2 (x_2 \partial_1^2) =
    x_2 \partial_1^2 \partial_2^2 - 2 \partial_1^2 \partial_2
    \parTheta{J} 2 \partial_1^2 \partial_2.
  \]
  This last term is a normal form for $S_{\Theta,J}$, so we get
  $\SThetaNF{J}(\partial_3^2 \D_2) = 2 \partial_1^2 \partial_2
  \neq 0$: $\Theta$ is not $J$-involutive. 
\end{enumerate}
\end{example}

The astute reader may remark that the last computation of the previous
example is closely related to the diagram appearing in 
Example~\ref{ex:diff_operators_rew}, which shows that $\rewTheta$ fails
to be confluent. This relationship between confluence and
$L$-involutivity is actually a very general one, as shown by the
following theorem.
\medskip

\begin{theorem}\label{thm:involutive_conf}
  Let $\Theta\subset\Weyl{n}$ be a finite set of differential operators,
  let $\prec$ be a monomial order on $\monBasis$ such that each element
  of $\Theta$ is monic, and let L be an involutive division 
  on~$\monBasis$ such that $\Theta$ is left L-autoreduced. If $\Theta$ is
  L-involutive, then the rewriting relation~$\rewTheta$ is confluent.
\end{theorem}

\begin{proof}
  Let $\SThetaL$ be the strategy for $\RTheta$ defined such as
  in~\eqref{equ:S-strategy}. Since rewriting rules of $\RTheta$ are of
  the form $\partial^\alpha\lm(\D)\rewTheta\partial^\alpha R(\D)$, where
  $\D\in\Theta$ and $\partial^\alpha\in\monBasis$, the assumption that
  $\Theta$ is $L$-involutive means that $\rewTheta$ is
  $\SThetaL$-confluent. By Theorem~\ref{thm:S-confluence_criterion},
  $\rewTheta$ is confluent.
\end{proof}
\smallskip

\begin{remark}
  As for term rewriting systems or \G\ bases theory, there exists a
  completion procedure in the situation of differential operators, which
  corresponds to Knuth-Bendix or Buchberger procedures. In the case of 
  the Janet example, it turns out that after a finite number of steps,
  this procedure yields the the following involutive set,
  see~\cite{MR1308976}:
  \[\overline{\Theta}=\left\{\D_1,\quad\D_2,\quad\partial_1^2\partial_2,
  \quad\partial_2^2\partial_3,\quad\partial_1^4,\quad\partial_1^2
  \partial_2\partial_3,\quad\partial_1^4\partial_3\right\}.\]
\end{remark}
\medskip

The end of this section aims to show that
axioms~\ref{it:div}--\ref{it:transitivity} in the definition of an
involutive division may be formulated in a purely rewriting language
using strategies. We fix a strategy~$S$ for $\rewTheta$. For every
$\D\in\Theta$, we say that $\lm(\D)$ {\em S-divides} the monomial
$\partial^\alpha\in\monBasis$ if $S$ contains a rewriting rule of the
form $\partial^\alpha\lm(\D)\parS\partial^\alpha r(\D)$ and we say that
the variable $\partial_i\in\Delta$ is {\em S-multiplicative} for $\D$ if
$\partial_i\lm(\D)$ is $S$-divisible by $\lm(\D)$. 
\smallskip

\begin{definition}
  A strategy $S$ for $\RTheta$ is said to be {\em involutive} if for
  every left-hand side $\partial^\alpha\lm(\D)$ of a rewriting rule in
  $S$, then $\partial^\alpha$ contains only $S$-multiplicative variables
  of $\D$.
\end{definition}
\smallskip

\begin{proposition}
  If the strategy $S$ is involutive, then the S-division satisfies
  axioms~\ref{it:div}--\ref{it:transitivity} of the definition of an
  involutive division. Moreover, if L is an involutive division on
  $\monBasis$ such that $\Theta$ is left L-autoreduced, then the
  $\SThetaL$-division is the restriction of L to $\lm(\Theta)$.
\end{proposition}

\begin{proof}
  Let us show the first assertion. Axioms~\ref{it:div}, \ref{it:vertex},
  and~\ref{it:transitivity} hold since $S$ is a strategy for $\RTheta$.
  Indeed, left-hand sides of $\RTheta$ are of the form
  $\partial^\alpha\lm(\D)$, hence~\ref{it:div}, and left-hand sides of
  elements of $S$ are pairwise distinct, hence~\ref{it:vertex}
  and~\ref{it:transitivity}. Moreover, axioms~\ref{it:unit}
  and~\ref{it:mul} hold by definition of an involutive strategy.

  Let us show the second assertion. By definition of the strategy
  $\SThetaL$ and of the $\SThetaL$-division, $\lm(\D)$ has the same set
  of multiplicative variables for $L$ and for the $\SThetaL$-division.
  Hence, a monomial~$\partial^\alpha$ is $L$-divisible by $\lm(\D)$, with
  $\D\in\Theta$, if and only if it is $\SThetaL$-divisible by $\lm(\D)$.
  That proves the assertion.
\end{proof}

\section{Conclusion and perspectives}

In this paper, we considered rewriting systems over vector spaces, where
we proposed an alternative approach to the traditional one, since we used
parallel rewriting steps. We also established some links with the
traditional approach, by giving a confluence criterion as well as a proof
of the diamond's lemma, based on strategies. Finally, we showed that our
general framework may be adapted to rational Weyl algebras, where
coefficients do not commute with monomials. In particular, we proved that
an involutive set in a rational Weyl algebra induces a confluent
rewriting system on it. We now present some possible extensions of our
work.
\medskip

A first research direction is to investigate the so-called
{\em standardisation properties}~\cite{Mellies05jwklop} associated to a
rewriting strategy. Indeed, the choice of strategy is nothing but the
choice for every vector of a preferred rewriting sequence starting at 
this vector. Moreover, we have shown in Point~\ref{it:rewS_transR} of
Proposition~\ref{prop:strategies} that each elementary rewriting step for
a strategy has same source and target points than a rewriting sequence
involving rules that do not belong to the strategy. The proof of this
fact is based on complete developments of residuals, that play a central
role in standardisation results, corresponding to left-hand sides of the
strategy. As a particular case, we expect to interpret Janet bases in
terms of standardisation.
\medskip

A second research direction is to extend our work to other algebraic
structures than vector spaces. This looks promising since we do away with
the notion of well-formed rewriting step, specific to the vector space
case. More generally, we hope to be able to extend our results to an
arbitrary category $\mathcal C$ (satisfying some suitable condition), 
recovering abstract rewriting in the case where $\mathcal C$ is the
category of sets, and linear rewriting as presented in this work in the
case where $\mathcal C$ is the category of vector spaces. Instead of
having a set or a vector space of terms to be rewritten, one would then
have an object of terms, which would be an object of $\mathcal C$. 
\medskip

A last research direction consists in applying rewriting systems over
rational Weyl algebras to the formal analysis of systems of partial
differential equations. As mentioned above, this topic covers many kinds
of problems and many techniques coming from rewriting theory and algebra
may be used in this context. We may focus on using rewriting methods
applied to the {\em Spencer cohomology}~\cite{MR1308976}, which, roughly
speaking, provides intrinsic properties, namely {\em $2$-acyclicity} and
{\em formal integrability}, that guarantee existence of normal form power
series solutions. 

\bibliography{Biblio}

\end{document}
